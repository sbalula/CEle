% ****** Start of file .tex ******
%
%   This file is based on apssamp.tex, part of the APS files in the REVTeX 4.1 distribution.
%   Version 4.1r of REVTeX, August 2010
%
%   Copyright (c) 2009, 2010 The American Physical Society.
%   Samuel Balula, Pedro Ribeiro, Luís Macedo, Eduardo Neto 2013
%   See the REVTeX 4 README file for restrictions and more information.
%
% TeX'ing this file requires that you have AMS-LaTeX 2.0 installed
% as well as the rest of the prerequisites for REVTeX 4.1
%
% See the REVTeX 4 README file
% It also requires running BibTeX. The commands are as follows:
%
%  1)  latex filename.tex
%  2)  bibtex filename
%  3)  latex filename.tex
%  4)  latex filename.tex

\documentclass[%
  reprint,
  %superscriptaddress,
  %groupedaddress,
  %unsortedaddress,
  %runinaddress,
  %frontmatterverbose, 
  %preprint,
  %showpacs,preprintnumbers,
  nofootinbib,
  %nobibnotes,
  %bibnotes,
  amsmath,amssymb,
  aps,
  %pra,
  %prb,
  %rmp,
  %prstab,
  %prstper,
  %floatfix,
  10pt,
]{revtex4-1}



\usepackage{facil}                      % Pacote pessoal
\usepackage{verbatim}                   % Apresentação de código
\usepackage{graphicx}                   % Include figure files
\usepackage{dcolumn}                    % Align table columns on decimal point
\usepackage{bm}                         % bold math
\usepackage[latin1,utf8]{inputenc}      % Tipos de caracteres
\usepackage[portuges]{babel}            % Português
\usepackage{indentfirst}                % Identação da primeira linha
\usepackage{hyperref}                   % add hypertext capabilities
\usepackage{float}                      %Fixar imagens
%\usepackage[mathlines]{lineno}          % Enable numbering of text and display math
%\linenumbers\relax                      % Commence numbering lines
%\usepackage[compact]{titlesec}

\usepackage[%showframe,%Uncomment any one of the following lines to test 
%%scale=0.7, marginratio={1:1, 2:3}, ignoreall, % default settings
%%text={7in,10in},centering,
margin=0.5in,        %diminuir margens
%total={6.5in,8.75in}, top=1.2in, left=0.9in, 
includefoot
%height=10in,a5paper,hmargin={3cm,0.8in},
]{geometry}

\begin{document}
\preprint{APS/123-QED}
%\captionsetup[table]{font=small,skip=0pt}
%\captionsetup[figure]{font=small,skip=0pt}
%\titlespacing{\section}{0pt}{*0}{*0}            %Poupar espaço
%\titlespacing{\subsection}{0pt}{*0}{*0}
%\titlespacing{\subsubsection}{0pt}{*0}{*0}


% % % % % % % % % % % % % % % % % % % % % % % % % % % % % % % % % % % % % % % % 
%%%%%%%%%%%%%%%%%%%%%%%%%%%%%%%%%% Início %%%%%%%%%%%%%%%%%%%%%%%%%%%%%%%%%%%%%%
% % % % % % % % % % % % % % % % % % % % % % % % % % % % % % % % % % % % % % % %
 

\title{}
%\thanks{}

\author{Pedro Ribeiro}%
\email{73221, pedro.q.ribeiro@tecnico.ulisboa.pt}
\author{Luis Macedo}%
\email{73633, luis.macedo@tecnico.ulisboa.pt}
\author{Samuel Balula}%
\email{72735, samuel.balula@tecnico.ulisboa.pt}

\affiliation{
  Instituto Superior Técnico\\
  Mestrado em Engenharia Física Tecnológica\\
  Complementos de Electrónica
}

%\collaboration{Grupo 57}

\date{\today}

%%%%%%%%%%%%%%%%%%%%%%%%%%%%%%%%%% Abstract %%%%%%%%%%%%%%%%%%%%%%%%%%%%%%%%%%%%
\begin{abstract}

\end{abstract}
\maketitle


%%%%%%%%%%%%%%%%%%%%%%%%%%%%%%%%%% Introdução %%%%%%%%%%%%%%%%%%%%%%%%%%%%%%%%%%
\section{Introdução}
\label{s:intro}
%Situar o problema, incluir fórmulas.
%Quem lê o relatório deve conseguir perceber exatamente o que foi feito.
\subsection*{I.2. Determinação da característica magnética do núcleo do transformador}
Conforme a nomenclatura da \rfig{../img/I-2_esq.png}, podem determinar-se os campos B e H em função dos valores de tensão medidos no osciloscópio.
\eq{\oint_l \mathbf{H} \cdot \, \mathbf{dl} = n_1 i_1}
\eq{H \approx \frac{n_1 i_1}{l} = \frac{n_1 Y_1}{R_1 l} }


\eq{u_2 = \frac{d \phi_2}{d t} \approx n_2 S \frac{d B}{dt}}
\eq{u_2 = uc + R C \frac {d u_c}{d _t} \approx R C \frac{d u_c}{dt}}
\eq{B \approx \frac{R_2 C_1}{n_2 S} Y_2}

%%%%%%%%%%%%%%%%%%%%%%%%%%%% Experiência realizada %%%%%%%%%%%%%%%%%%%%%%%%%%%%%
\section{Experiência Realizada}
\label{s:expreal}
%Incluir diagrama de blocos da montagem
Apresentam-se na tabela \ref{partlist} os componentes utilizados.

\tabela[partlist]{Lista dos componentes utilizados}{lrrr}{
	Descrição		&Modelo/Valor		&Qt.			&Referência	\\ \hline
	Ampl. operacional	&			&1			&		\\
	Cond. cerâmico		&nF			&1			&		\\
	Resistência .25W	&$\Omega$		&4			&		\\
}

\fig{../img/I-2_esq.png}{I.2. Circuito utilizado na determinação da característica magnética do núcleo do transformador}

%%%%%%%%%%%%%%%%%%%%%%%%%%%%%%%% Resultados %%%%%%%%%%%%%%%%%%%%%%%%%%%%%%%%%%%%
\section{Resultados}
\label{s:resul}
%incluir tabelas. Excesso de dados => grafico com valores mais significativos

\subsection*{I.2. Determinação da característica magnética do núcleo do transformador}

\fig{../dat/ALL0000/A0000DS.png}{Característica Y2(Y1) do núcleo do transformador. Não foi possível converter os dados de tensão para unidades de campo por os dados exportados pelo osciloscópio não serem coerentes com a figura observada, não sendo portanto válidos}
\tabela{Parâmetros do núcleo do transformador}{ccc}{
Parâmetro	&Valor			&Unidades \\ \hline
comprimento	&$8.70\pm0.05$		&cm		\\
largura		&$3.95\pm0.05$		&cm		\\
altura		&$10.50\pm0.05$		&cm		\\
espessura	&$1.70\pm0.05$		&cm		\\	\hline
+B saturação	&$2.976\pm.041$		&T		\\
-B saturação	&$-2.935\pm0.041$	&T		\\
+B remanescente	&$1.447\pm.041$		&T		\\
-B remanescente	&$-1.323\pm.041$	&T		\\
+H coercivo	&$128.5\pm1.2$		&$A m^{-1}$	\\
-H coercivo	&$-134.4\pm1.2$		&$A m^{-1}$	\\
$\mu$ increm.	&$1.81\pm0.09$		&$T m A^{-1}$	\\
}


%%%%%%%%%%%%%%%%%%%%%%%%%%% Análise dos resultados %%%%%%%%%%%%%%%%%%%%%%%%%%%%%
\section{Análise de resultados}
\label{s:aresul}

\subsection*{I.2. Determinação da característica magnética do núcleo do transformador}
O transformador encontra-se bem dimensionado para a tensão normal de trabalho (230V), já que para estas tensões a sua característica B(H) não se apresenta na região de saturação.


%%%%%%%%%%%%%%%%%%%%%%%%%%% Conclusões e Críticas %%%%%%%%%%%%%%%%%%%%%%%%%%%%%%
\section{Conclusões e Críticas}
\label{s:conclu}
%Incluir melhorias propostas à experiência


%\begin{acknowledgments}
%\end{acknowledgments}

%%%%%%%%%%%%%%%%%%%%%%%%%%%%%%%%%%%%%%%%%%%%%%%%%%%%%%%%%%%%%%%%%%%%%%%%%%%%%%%%
% % % % % % % % % % % % % % % %     FIM    % % % % % % % % % % % % % % % % % % % 
%%%%%%%%%%%%%%%%%%%%%%%%%%%%%%%%%%%%%%%%%%%%%%%%%%%%%%%%%%%%%%%%%%%%%%%%%%%%%%%%

\nocite{*}
\bibliography{bibliografia}{}
\bibliographystyle{plain}% Produces the bibliography via BibTeX.
\end{document}
%end of file

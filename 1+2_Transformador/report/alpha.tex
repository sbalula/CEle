% ****** Start of file .tex ******
%
%   This file is based on apssamp.tex, part of the APS files in the REVTeX 4.1 distribution.
%   Version 4.1r of REVTeX, August 2010
%
%   Copyright (c) 2009, 2010 The American Physical Society.
%   Samuel Balula, Pedro Ribeiro, Luís Macedo, Eduardo Neto 2013
%   See the REVTeX 4 README file for restrictions and more information.
%
% TeX'ing this file requires that you have AMS-LaTeX 2.0 installed
% as well as the rest of the prerequisites for REVTeX 4.1
%
% See the REVTeX 4 README file
% It also requires running BibTeX. The commands are as follows:
%
%  1)  latex filename.tex
%  2)  bibtex filename
%  3)  latex filename.tex
%  4)  latex filename.tex

\documentclass[%
  reprint,
  %superscriptaddress,
  %groupedaddress,
  %unsortedaddress,
  %runinaddress,
  %frontmatterverbose, 
  %preprint,
  %showpacs,preprintnumbers,
  nofootinbib,
  %nobibnotes,
  %bibnotes,
  amsmath,amssymb,
  aps,
  %pra,
  %prb,
  %rmp,
  %prstab,
  %prstper,
  %floatfix,
  10pt,
]{revtex4-1}



\usepackage{facil}                      % Pacote pessoal
\usepackage{verbatim}                   % Apresentação de código
\usepackage{graphicx}                   % Include figure files
\usepackage{dcolumn}                    % Align table columns on decimal point
\usepackage{bm}                         % bold math
\usepackage[latin1,utf8]{inputenc}      % Tipos de caracteres
\usepackage[portuges]{babel}            % Português
\usepackage{indentfirst}                % Identação da primeira linha
\usepackage{hyperref}                   % add hypertext capabilities
\usepackage{float}                      %Fixar imagens
%\usepackage[mathlines]{lineno}          % Enable numbering of text and display math
%\linenumbers\relax                      % Commence numbering lines
%\usepackage[compact]{titlesec}

\usepackage[%showframe,%Uncomment any one of the following lines to test 
%%scale=0.7, marginratio={1:1, 2:3}, ignoreall, % default settings
%%text={7in,10in},centering,
margin=0.5in,        %diminuir margens
%total={6.5in,8.75in}, top=1.2in, left=0.9in, 
includefoot
%height=10in,a5paper,hmargin={3cm,0.8in},
]{geometry}

\begin{document}
\preprint{APS/123-QED}
%\captionsetup[table]{font=small,skip=0pt}
%\captionsetup[figure]{font=small,skip=0pt}
%\titlespacing{\section}{0pt}{*0}{*0}            %Poupar espaço
%\titlespacing{\subsection}{0pt}{*0}{*0}
%\titlespacing{\subsubsection}{0pt}{*0}{*0}


% % % % % % % % % % % % % % % % % % % % % % % % % % % % % % % % % % % % % % % % 
%%%%%%%%%%%%%%%%%%%%%%%%%%%%%%%%%% Início %%%%%%%%%%%%%%%%%%%%%%%%%%%%%%%%%%%%%%
% % % % % % % % % % % % % % % % % % % % % % % % % % % % % % % % % % % % % % % %
 

\title{Caracterização de um transformador monofásico}
%\thanks{}

\author{Pedro Ribeiro}%
\email{73221, pedro.q.ribeiro@tecnico.ulisboa.pt}
\author{Luis Macedo}%
\email{73633, luis.macedo@tecnico.ulisboa.pt}
\author{Samuel Balula}%
\email{72735, samuel.balula@tecnico.ulisboa.pt}

\affiliation{
  Instituto Superior Técnico\\
  Mestrado em Engenharia Física Tecnológica\\
  Complementos de Electrónica
}

%\collaboration{Grupo 57}

\date{\today}

%%%%%%%%%%%%%%%%%%%%%%%%%%%%%%%%%% Abstract %%%%%%%%%%%%%%%%%%%%%%%%%%%%%%%%%%%%
\begin{abstract}
Neste trabalho laboratorial determinam-se as características eléctricas do transformador monofásico, analisam-se as propriedades magnéticas do material do núcleo e analisa-se o comportamento do transformador em carga. Os parâmetros, do modelo de Steinmetz, são determinados por ensaios em vazio e curto circuito.
\end{abstract}
\maketitle


%%%%%%%%%%%%%%%%%%%%%%%%%%%%%%%%%% Introdução %%%%%%%%%%%%%%%%%%%%%%%%%%%%%%%%%%
\section{Introdução}
\label{s:intro}
%Situar o problema, incluir fórmulas.
%Quem lê o relatório deve conseguir perceber exatamente o que foi feito.

\subsection*{I.1./II Relações entre o primário e o secundário}
Conhecem-se as seguintes relações para um transformador ideal:
\begin{equation}
\frac{U_{pri}}{U_{sec}}=\frac{n_{pri}}{n_{sec}}
\label{eq:trans_volt}
\end{equation}
\begin{equation}
n_{pri}I_{pri}=n_{sec}I_{sec} \Rightarrow \frac{I_{sec}}{I_{pri}}=\frac{n_{pri}}{n_{sec}}
\label{eq:trans_corr}
\end{equation}
No entanto, os transformadores com núcleo de ferro possuem não-idealidades, pelo que se utiliza o esquema equivalente de Steinmetz, que pode ser visto na figura \ref{../img/I-3_steinmetz.png}, para o modelizar. Para estudar este tipo de transformador vamos supor duas condições: uma em que o secundário se encontra em vazio (isto é $R_{\mathrm{LOAD}}=\infty$) e outra em que o secundário já possui uma resistência de carga bem definida no secundário.\\
Para o primeiro caso, tem-se a seguinte condição:
\begin{equation}
i_2=i_2'=0
\end{equation}
Logo se não passa corrente em $i_2'$, toda a corrente irá passar pela resistência $r_{fe}$ e pela bobine $l_11\cos \phi_{fe}$. Logo:
\begin{equation}
\overline{U_{\mathrm{s}}}=\overline{I_{\mathrm{s}}}(r_1+r_{fe}+j\omega(l_{11}\cos \phi_{fe}+\lambda_{11}))
\end{equation}
\begin{equation}
\overline{U_2'}=\overline{I_{\mathrm{s}}} (r_{fe}+j\omega l_{11}\cos \phi_{fe})
\end{equation}
A partir destas equações, tem-se:
\begin{equation}
\overline{U_2}=\frac{n_{sec}}{n_{pri}}\frac{r_{fe}+j\omega l_{11}\cos \phi_{fe}}{r_1+r_{fe}+j\omega(l11\cos \phi_{fe}+\lambda_{11})}\overline{U_\mathrm{s}}
\label{eq:aberto}
\end{equation}
Para o caso em que o enrolamento secundário se encontra fechado através de uma resistência de carga ($R_{\mathrm{LOAD}}$), visto que $r_{fe}\gg r_2$ e que vai existir agora corrente a passar em $i_2'$, pode-se então admitir:
\begin{equation}
i_{r_{fe}}\ll i_2 \Rightarrow -i_2'\approx i_1=i,\ \ i_{r_{fe}}\approx 0
\end{equation}
Utilizando esta aproximação, tem-se, determinando a equação de circulação no primário:
\begin{equation}
\overline{U_{\mathrm{s}}}=\overline{I_{\mathrm{s}}}(r_1+r_2'+j\omega(\lambda_{11}+\lambda_{22}')+\overline{U_2'}
\label{eq:cc}
\end{equation}
Para conseguir expressar $\overline{U_2}=\overline{U_2}(\overline{U_{\mathrm{s}}})$ é necessário determinar qual a impedância na bobine do primário. Recorrendo às equações \ref{eq:trans_volt} e \ref{eq:trans_corr} e à lei de Ohm, chega-se a:
\begin{equation}
\frac{\overline{U_{2}'}}{\overline{U_{2}}}=\frac{n_{pri}}{n_{sec}}\Rightarrow \frac{Z_2'}{Z_2}=\left (\frac{n_{pri}}{n_{sec}}\right)^2
\end{equation}
Visto que $Z_2=R_{\mathrm{LOAD}}$, pode-se escrever \ref{eq:cc} como:
\begin{equation}
\overline{U_{\mathrm{s}}}=\overline{I_{\mathrm{s}}}(r_1+r_2'+R_{\mathrm{LOAD}}\left(\frac{n_{pri}}{n_{sec}}\right)^2+j\omega(\lambda_{11}+\lambda_{22}')
\label{eq:tens_u2}
\end{equation}
Da mesma maneira que no exercício anterior, fazendo:
\begin{equation}
\overline{I_s}=\frac{U_2'}{R_{LOAD}\left(\frac{n_{pri}}{n_{sec}}\right)^2}
\end{equation}
Chega-se à relação final da tensão em carga:
\begin{equation}
\overline{U_2}=\frac{\left(\frac{n_{pri}}{n_{sec}}\right)R_{\mathrm{LOAD}}}{(r_1+r_2'+R_{\mathrm{LOAD}}\left(\frac{n_{pri}}{n_{sec}}\right)^2+j\omega(\lambda_{11}+\lambda_{22}')}\overline{U_s}
\label{eq:derradeira}
\end{equation}
Recorrendo à expressão \ref{eq:tens_u2}, sabendo que $\overline{U_2}=R_{LOAD}\overline{I_2}$ e substituindo em \label{eq:derradeira}, obtém-se a relação de correntes:
\begin{equation}
\overline{I_2}=\left(\frac{n_{pri}}{n_{sec}}\right)\overline{I_s}
\label{eq:cool}
\end{equation}
A relação de impedâncias é dada por:
\begin{equation}
Z_s=Z_{cc}+\left (\frac{n_{pri}}{n_{sec}}\right)^2Z_2
\end{equation}
Em que $Z_2=R_{\mathrm{LOAD}}$ e
\begin{equation}
Z_{cc}=r_1+r_2'+j\omega(\lambda_{11}+\lambda_{22}')
\end{equation}
Por fim a eficiência pode ser calculada fazendo:
\begin{equation}
\eta=\frac{P_{20}}{P_{10}}=\frac{\mathrm{Re}(U_2I_2)}{\mathrm{Re}(U_1I_1)}
\end{equation}
Através da lei de ohm e das expressões \ref{eq:tens_u2} e \ref{eq:cool}, obtém-se finalmente:
\begin{equation}
\eta=\frac{\left(\frac{n_{pri}}{n_{sec}}\right)^2R_\mathrm{LOAD}}{r_1+r_2'+R_{\mathrm{LOAD}}\left(\frac{n_{pri}}{n_{sec}}\right)^2}
\end{equation}


\subsection*{I.2. Determinação da característica magnética do núcleo do transformador}
Conforme a nomenclatura da \rfig{../img/I-2_esq.png}, podem determinar-se os campos B e H em função dos valores de tensão medidos no osciloscópio.
\eq{\oint_l \mathbf{H} \cdot \, \mathbf{dl} = n_1 i_1}
\eq{H \approx \frac{n_1 i_1}{l} = \frac{n_1 Y_1}{R_1 l} }


\eq{u_2 = \frac{d \phi_2}{d t} \approx n_2 S \frac{d B}{dt}}
\eq{u_2 = uc + R C \frac {d u_c}{d _t} \approx R C \frac{d u_c}{dt}}
\eq{B \approx \frac{R_2 C_1}{n_2 S} Y_2}

\subsection*{I.3. Ensaio em vazio do transformador}
O objectivo deste ponto é determinar as impedâncias internas do transformador. A representação do transformador de acordo com o esquema equivalente de Steinmetz encontra-se na figura \rfig{../img/I-3_steinmetz.png}.
Começando por estudar o secundário em vazio é possível determinar $r_1$,$\lambda_{11}$, $r_{fe}$, $l_{11}*cos(\phi_{fe})$, a partir de $I_{1ef}$, $U_{2ef}$, $P_{10}=<u_1 i_1>$ e $P_{12}=<u_2 i_1>$ utilizando as seguintes relações

\eq{r_{fe}=\frac{P_{12}}{\frac{n_2}{n_1}I_{1ef}^2}}

\eq{r1=\frac{P_{10}}{I_{1ef}^2}-r_{fe}}

\eq{l_{11}=\frac{U_{2ef}}{I_{1ef}\frac{n2}{n1}\omega}}

\eq{\phi_{fe}=Arcsin{\frac{P_{12}}{U_{2ef}{I_{1ef}}}}}

\eq{\lambda_{11}=\sqrt{\frac{\left(\frac{U_{1ef}}{I_{1ef}}\right)^2-(r_1+r_{fe})^2}{\omega^2}}-l_{11}cos(\phi_{fe})}

Estudando agora o primário em vazio é possível determinar $r_2'$,$\lambda_{22}'$, $r_{fe}$, $l_{11}cos(\phi_{fe})$ a partir de $I_{2ef}$, $U_{1ef}$,$P_{2}=<u_2 i_2>$ e $P_{21}=<u_1 i_2>$, utilizando as relações

\eq{r_{fe}=\frac{P_{21}}{I_{2ef}^2\frac{n2}{n1}}}

\eq{r_{2}'=\frac{P_2}{\left(I_{2ef}\frac{n_2}{n_1}\right)^2}-r_{fe}}

\eq{l_{11}=\frac{U_{1ef}}{\frac{n_2}{n_1}I_{2ef}\omega}}

\eq{\phi_{fe}=Arcsin{\frac{r_{fe}}{\omega l_{11}}}}

\eq{\lambda_{22}'=\sqrt{\frac{\left(\frac{U_{2ef}}{I_{1ef}}\right)^2\frac{1}{\left(\frac{n_2}{n_1}\right)^4}-(r_{2}'+r_{fe})^2}{\omega^2}}-l_{11}cos(\phi_{fe})}

\fig{../img/I-3_steinmetz.png}{I.3. Esquema equivalente de Steinmetz}

\subsection*{I.4. Ensaio em curto-circuito do transformador}
Neste estudo do transformador com o secundário em curto-circuito assume-se que nenhuma corrente passa por $r_{fe}$ e $l_{11}cos(\phi_{fe})$. Neste caso é possível determinar $r_1+r_{2'}$ e $\lambda_{11} + \lambda_{22}'$ a partir de $U_{1ef}$ e de $P_{10}=<u_1 i_1>$ utilizando as seguintes relações

\eq{r_1+r_2'=\frac{P_{10}}{I_{1ef}^2}}

\eq{\lambda_{11} + \lambda_{22}'=\sqrt{\frac{\left(\frac{U_{1ef}}{I_{1ef}}\right)^2-(r_1+r_2')^2}{\omega^2}}}



%%%%%%%%%%%%%%%%%%%%%%%%%%%% Experiência realizada %%%%%%%%%%%%%%%%%%%%%%%%%%%%%
\section{Experiência Realizada}
\label{s:expreal}
%Incluir diagrama de blocos da montagem
Apresentam-se na tabela \ref{partlist} os componentes utilizados.

\tabela[partlist]{Lista dos componentes utilizados}{lrr}{
	Descrição		&Modelo/Valor		&Qt.				\\ \hline
	Transformador		&			&1	\\
	Cond. cerâmico		&1$\mu$F		&1	\\
	Resistência 10W		&1$\Omega$		&1	\\
	Resistência 10W		&10$\Omega$		&2	\\
	Resistência 2.5W	&10K$\Omega$		&4	\\
	Potenciometro		&12$\Omega$		&1	\\
		
}

\fig{../img/I-2_esq.png}{I.2. Circuito utilizado na determinação da característica magnética do núcleo do transformador}


Para se fazer o estudo do transformador com o secundário em vazio utilizou-se o circuito representado na figura \ref{../img/I-3_secundario_vazio.png}
\fig{../img/I-3_secundario_vazio.png}{I.3. Circuito utilizado no estudo do transformador com o secundário em vazio}

Utilizaram-se as duas resistências de $10K\Omega$ para que fosse possível medir a tensão no osciloscópio, uma vez que se aplicaram cerca de 220V.

Para estudar o transformador com o primário em vazio e o transformador com o secundário em curto circuito utilizaram-se os circuitos representados nas figuras \ref{../img/I-3_primario_vazio.png} e \ref{../img/I-4_curto_circuito.png}, respectivamente. 
\fig{../img/I-3_primario_vazio.png}{I.3. Circuito utilizado no estudo do transformador com o primário em vazio}

\fig{../img/I-4_curto_circuito.png}{I.4. Circuito utilizado no estudo do transformador com o secundário em curto circuito}


%%%%%%%%%%%%%%%%%%%%%%%%%%%%%%%% Resultados %%%%%%%%%%%%%%%%%%%%%%%%%%%%%%%%%%%%
\section{Resultados}
\label{s:resul}
%incluir tabelas. Excesso de dados => grafico com valores mais significativos

\subsection*{I.2. Determinação da característica magnética do núcleo do transformador}
Apresentam-se na \rfig{../dat/ALL0000/A0000DS.png} e na tabela \ref{hujf} os resultados experimentais.
\fig{../dat/ALL0000/A0000DS.png}{Característica Y2(Y1) do núcleo do transformador. Não foi possível converter os dados de tensão para unidades de campo por os dados exportados pelo osciloscópio não serem coerentes com a figura observada, não sendo portanto válidos}
\tabela[hujf]{Parâmetros do núcleo do transformador}{ccc}{
Parâmetro	&Valor			&Unidades \\ \hline
comprimento	&$8.70\pm0.05$		&cm		\\
largura		&$3.95\pm0.05$		&cm		\\
altura		&$10.50\pm0.05$		&cm		\\
espessura	&$1.70\pm0.05$		&cm		\\	\hline
+B saturação	&$2.976\pm.041$		&T		\\
-B saturação	&$-2.935\pm0.041$	&T		\\
+B remanescente	&$1.447\pm.041$		&T		\\
-B remanescente	&$-1.323\pm.041$	&T		\\
+H coercivo	&$128.5\pm1.2$		&$A m^{-1}$	\\
-H coercivo	&$-134.4\pm1.2$		&$A m^{-1}$	\\
$\mu$ increm.	&$1.81\pm0.09$		&$T m A^{-1}$	\\
}

\subsection*{I.3. Ensaio em vazio do transformador}
Neste ponto considerou-se que o erro das grandezas dadas pelo software de aquisição era de $0,5\%$ do valor da grandeza. Os resultados fornecidos pelo software para o estudo do transformador com o secundário e primário em vazio estão representados nas tabelas \ref{tab:dadosvaziosec} e \ref{tab:dadosvaziopri}.
  \tabela[tab:dadosvaziosec]{Dados recolhidos do software para o estudo do transformador com o secundário em vazio}{c|c|c}{
    Posição & a     & b \\
  
    $V1_ef(ac) (V)$ & $218,139\pm1,091$ & $25,705\pm0,129$ \\ \hline
    $V2_ef(ac)(V)$ & $0,840\pm0,004$ & $0,857\pm0,004$ \\ \hline
    $<v1*v2>ac(V^2)$ & $54,555\pm0,273$ & $-6,475\pm0,032$ \\ \hline
    $<v1>(V)$ & $-5,609\pm0,028$ & $-0,059\pm0,001$ \\ \hline
    $<v2>(V)$ & $2,989\pm0,015$ & $0,0248\pm0,001$ \\     
}
  \tabela[tab:dadosvaziopri]{Dados recolhidos do software para o estudo do transformador com o primário em vazio}{c|c|c}{
    Posição & a     & b \\
  
    $V1_ef(ac) (V)$ & $219,047\pm1,095$ & $28,983\pm0,145$ \\ \hline
    $V2_ef(ac)(V)$ & $7,202\pm0,036$ & $7,069\pm0,035$ \\ \hline
    $<v1*v2>ac(V^2)$ & $52,047\pm0,260$ & $110,506\pm0,552$ \\ \hline
    $<v1>(V)$ & $-4,150\pm0,021$ & $-0,025\pm0,001$ \\ \hline
    $<v2>(V)$ & $0,175\pm0,001$ & $0,236\pm0,001$ \\     
}
Sendo que a posição $a$ corresponde à ponta de prova do osciloscópio a medir a tensão no primário e a posição $b$ corresponde à ponta de prova do osciloscópio a medir a tensão no secundário.
É de notar que era pretendido obter o mesmo estado de magnetização para ambas as situações, sendo que para tal, aquando do estudo do transformador com o primário em vazio, se impôs uma tensão no secundário semelhante à tensão obtida no secundário em vazio, imponto no primário cerca de $220V$
\subsection*{I.4. Ensaio em curto-circuito do transformador}

Os dados obtidos pelo software para o estudo do transformador em curto-circuito encontram-se na tabela \ref{tab:dadoscc}

 \tabela[tab:dadoscc]{Dados recolhidos do software para o estudo do transformador em curto-circuito}{c|c}{   
  
    $V1_ef(ac) (V)$ & $33,403\pm0,167$  \\ \hline
    $V2_ef(ac)(V)$ & $0,837\pm0,004$  \\ \hline
    $<v1*v2>ac(V^2)$ & $21,176\pm0,106$  \\ \hline
    $<v1>(V)$ & $-0,240\pm0,001$ \\ \hline
    $<v2>(V)$ & $0,024\pm0,001$ \\
} 
Neste ponto impôs-se uma corrente de cerca de $I_{1ef}=820mA$ visto que a corrente nominal do transformador é de $818mA$.

%%%%%%%%%%%%%%%%%%%%%%%%%%% Análise dos resultados %%%%%%%%%%%%%%%%%%%%%%%%%%%%%
\section{Análise de resultados}
\label{s:aresul}
\subsection*{I.1. Determinação da relação de espiras}
A relação entre o número de espiras de dois enrolamentos e as suas tensões é linear, pelo que se pode ajustar uma função do tipo:
\begin{equation}
y=ax+b
\end{equation}
Sendo que a vai representar a relação entre espiras. Para determinar a relação entre espiras do primário e do secundário e o número aproximado de espiras em cada enrolamento, fazendo relação com um enrolamento com um número conhecido de espiras.
Efectuando o ajuste linear aos dados obtidos obtiveram-se os gráficos \ref{../dat/p_vs_s.pdf}, \ref{../dat/5_vs_prim.pdf} e \ref{../dat/5_vs_sec.pdf}.
\fig{../dat/p_vs_s.pdf}{Ajuste obtido para a relação entre primário e secundário}
Para a relação entre primário e secundário obteve-se $a=\frac{n_{pri}}{n_{sec}}=8.63\pm0.01$.\\
\fig{../dat/5_vs_prim.pdf}{Ajuste obtido para a relação entre primário e o enrolamento de 5 espiras}
Para a relação entre o enrolamento de 5 espiras e o enrolamento primário obteve-se $a=\frac{n_{5}}{n_{pri}}=0.008\pm0.001$, o que implica que o número de espiras no enrolamento primário deverá ser $n_{pri}=621.89\pm0.01$.\\
\fig{../dat/5_vs_sec.pdf}{Ajuste obtido para a relação entre o secundário e o enrolamento de 5 espiras}
Para a relação entre o enrolamento de 5 espiras e o enrolamento secundário obteve-se  $a=\frac{n_{5}}{n_{sec}}=0.0694\pm0.001$ o que implica que o número de espiras no enrolamento secundário deverá ser $n_{sec}=72.05\pm0.01$.\\
Para confirmar, pode-se dividir o valor obtido através do ajuste com o número de espiras do secundário e do primário, o que resulta em:
\begin{equation}
\frac{n_{pri}}{n_{sec}}=\frac{621.89}{72.05}=8.63\approx a_{\mathrm{ajuste}}
\end{equation}
 O que implica que o valor da relação de espiras foi bem determinado.

\subsection*{I.2. Determinação da característica magnética do núcleo do transformador}
O transformador encontra-se bem dimensionado para a tensão normal de trabalho (230V), já que para estas tensões a sua característica B(H) não se apresenta na região de saturação.


\subsection*{I.3. Ensaio em vazio do transformador}
Os dados obtidos para o estudo do transformador em vazio encontram-se na tabela \ref{tab:vazio}, sendo que a situação A corresponde ao transformador com o secundário em vazio e a situação B corresponde ao transformador com o primário em vazio.

 \tabela[tab:vazio]{Dados obtidos para o estudo do transformador em vazio}{c|c|c}{
          Casos                    & A   & B \\
  
    $I_{1ef} (A)$ & $0,084\pm0,001$ & --- \\ \hline
    $ I_{2ef} (A)$ & --- & $0,720\pm0,003$ \\ \hline
    $P_{12}  (W)$ & $0,647\pm0,003$ & --- \\ \hline
    $ P_{21} (W)$ &--- & $5,205\pm0,026$ \\ \hline
    $P_{10}  (W)$ & $5,455\pm0,027$ & $11,051\pm0,055$ \\ \hline
    $ P_{20} (W)$&---& $11,051\pm0,055$ \\ \hline
    $r_{fe}(\Omega)$ & $791,982\pm11,959$ & $86,609\pm1,308$ \\ \hline
    $r_1   (\Omega)$ & $-18,813\pm23,556$ & --- \\  \hline
    $ r_2'(\Omega)$ & --- & $1500,465\pm25,433$ \\ \hline
    $\phi_{fe}$ & $17,449\pm0,005$ & $1,891\pm0,001$ \\ \hline
    $cos(\phi_{fe})$ & $0,953\pm0,001$ & $0,999\pm0,001$ \\ \hline
    $l_{11}(H)$ & $8,406\pm0,085$ & $8,356\pm0,084$ \\ \hline
    $\lambda_{11} (H)$ & $-0,129\pm8,949$ &  \\ \hline
    $\lambda_{22}'(H)$ & --- & $-0,257\pm0,360$\\  
}

Note-se que para estes cálculos se utilizou $I_{ef}=	\frac{U_R}{R}$, sendo $U_R$ a tensão aos terminais da resistência $R=10\Omega$.
O pretendido neste ponto era obter um estado de magnetização semelhante com o secundário em aberto e com o primário em aberto. Isso é alcançado se os valores de $r_{fe}$, $\phi_{fe}$  e $l_{11}$ forem semelhantes. Analisando os resultados obtidos conclui-se que os valores de $l_{11}$ são bastante próximos mas que todos os outros valores apresentam uma grande discrepância. Dado que o valor dos parâmetros do esquema equivalente de Steinmetz variam com as tensões aplicadas e com o estado de magnetização, em particular $l_{11}$, e tendo em conta  que os valores de $l_{11}$ obtidos são semelhantes conclui-se que os estados de magnetização para ambas as situações são semelhantes.
Note-se que alguns parâmetros têm valores negativos, mas os erros englobam valores positivos, pelo que uma interpretação possível é que os valores sejam tão pequenos que se tenham obtido valores negativos.


Pode-se comparar os dados obtidos com a previsão teórica, fazendo o módulo da fórmula \ref{eq:aberto}. Os resultados obtidos estão na tabela \ref{tab:vazio}.
\begin{table}[h]
\begin{tabular}{|c|c|}
                       & Tensão (V)     \\ \hline
$V1_{ef}$  imposto     & 219$\pm$1      \\
$V2_{ef}$ teórico      & 25.29$\pm$2.67 \\
Desvio à exactidão (\%) & 14.5\%        
\end{tabular}
\caption{Valores obtidos teoricamente para o secundário em vazio.}
\label{tab:vazio}
\end{table}

\subsection*{I.4. Ensaio em curto-circuito do transformador}
Os dados obtidos para o estudo do transformador em curto-circuito encontram-se na tabela \ref{tab:tabelao}. Os valores de corrente, impedância e eficiência apresentam-se na tabela \ref{tab:imp} Utilizando o módulo das relações obtidas nas fórmulas presentes na introdução teórica, pode-se prever teoricamente as tensões, correntes e impedâncias no secundário. Utilizou-se como tensão no primário a tensão medida no multímetro, visto esta ser mais precisa que a obtida através do osciloscópio. Os resultados apresentam-se na tabela \ref{tab:carga}.

\begin{table}[h]
\begin{tabular}{c|c|c|c}
                         & R=20$\Omega$  & R=15$\Omega$  & R=10$\Omega$  \\ \hline
$I2_{ef}$ experim. (A)   & 1.25$\pm$0.01 & 1.66$\pm$0.01 & 2.45$\pm$0.01 \\ \hline
$Z_s$ teórico ($\Omega$) & 1230$\pm$19   & 1080$\pm$15   & 680$\pm$6     \\ \hline
Eficiência (\%)          & 88.54         & 73.13         & 41.57        
\end{tabular}
\caption{Valores obtidos experimentalmente para a eficiência, corrente no secundário e impedância}
\label{tab:imp}
\end{table}

\begin{table}[h]
\begin{tabular}{c|c|c|c}
                         & R=10$\Omega$    & R=15$\Omega$    & R=20$\Omega$    \\ \hline
$V1_{ef}$ imposta (V)    & 220.0$\pm$1.1   & 222.6$\pm$1.1   & 222.6$\pm$1.1   \\ \hline
$V2_{ef}$ teorica (V)    & 24.81$\pm$0.27  & 25.14$\pm$0.27  & 25.40$\pm$0.27  \\ \hline
$I2_{ef}$ teórica (V)    & 2.851$\pm$0.089 & 1.803$\pm$0.089 & 1.572$\pm$0.088 \\ \hline
$Z_s$ teórico ($\Omega$) & 774$\pm$2       & 1146$\pm$3      & 1518$\pm$4     \\ \hline
$\eta$ teórico (\%) &96.2&97.5&98
\end{tabular}
\caption{Valores obtidos para a previsão teórica das tensões, correntes e impedâncias}
\label{tab:carga}
\end{table}
Os desvios à exactidão dos vários valores experimentais estão apresentados na tabela \ref{tab:exac}
\begin{table}[h]
\begin{tabular}{c|c|c|c}
Desvio à exactidão (\%) & R=10$\Omega$ & R=15$\Omega$ & R=20$\Omega$ \\ \hline
$V2_{ef}$               & 1.3          & 0.7          & 1.3          \\ \hline
$I2_{ef}$               & 14           & 8            & 17           \\ \hline
$Z_s$                   & 12           & 6            & 19           \\ \hline
$\eta$                 &9           &25           &57
\end{tabular}
\caption{Desvios à exactidão das várias grandezas do transformador obtidas}
\label{tab:exac}
\end{table}

\subsection*{II.Ensaio em carga do transformador}
Os dados obtidos para várias resistência de carga apresentam-se na tabela \ref{tab:carga}

\tabelao[tab:tabelao]{Dados obtidos para o transformador em carga}{c|c|c|c|c|c|c}{  
                             & \multicolumn{2}{c|}{R=20$\Omega$}     & \multicolumn{2}{c|}{R=15$\Omega$}   & \multicolumn{2}{c|}{R=10$\Omega$}    \\ \cline{2-7} 
                               & Primário           & Secundário       & Primário         & Secundário       & Primário          & Secundário       \\ \hline
V1ef(ac)                       & 223,517$\pm$1.11   & 25,01$\pm$0.13   & 225,693$\pm$1.12 & 24,957$\pm$0.12  & 225,367$\pm$1.12  & 24,474$\pm$0.12  \\ \hline
V2ef(ac)                       & 1,817$\pm$0.009    & 1,821$\pm$0.009  & 2,089$\pm$0.010  & 2,086$\pm$0.010  & 3,314$\pm$0.017   & 3,314$\pm$0.017  \\ \hline
\textless v1xv2\textgreater(ac) & 353,221$\pm$12.611 & 39,659$\pm$0.183 & 425,83$\pm$2.129 & 47,222$\pm$0.236 & 720,513$\pm$3.603 & 78,67$\pm$0.393  \\ \hline
\textless v1\textgreater        & -4,4$\pm$0.1       & -0,12$\pm$0.01   & -3,3$\pm$0.01    & -0,112$\pm$0.001 & -3,97$\pm$0.01    & -0.093$\pm$0.002 \\ \hline
\textless v2\textgreater        & 0.037$\pm$0.01          &4.56$\pm$0.01        & 0.044$\pm$0.01        & 0.052$\pm$0.01        & 0.054$\pm$0.01         & 0.067$\pm$0.01        \\ \hline
multimetro                     & 220$\pm$1.1        & 220$\pm$1.1      & 222,6$\pm$1.1    & 222,3$\pm$1.1    & 222.6$\pm$1.1     & 222,6$\pm$1.1}


\subsection*{I.4. Ensaio em curto-circuito do transformador}
Os dados obtidos para o estudo do transformador em curto-circuito encontram-se na tabela \ref{tab:curtocircuito}

\tabela[tab:curtocircuito]{Dados recolhidos do software para o estudo do transformador em curto-circuito}{c|c}{   
    $I_{1ef}(A)$ & $0,837\pm0,004$  \\ \hline
    $U_{2ef}(ac)(V)$ & $32,774\pm0,164$  \\ \hline
    $P_{10}(V)$ & $20,339\pm0,101$  \\ \hline
    $r_1+r_2'(\Omega)$ & $29,026\pm0,435$ \\ \hline
    $\lambda_{11}+\lambda_{22}'(H)$ & $0,084\pm0,003$ \\
}

Comparando os resultados obtidos no ponto $4$ com os obtidos no ponto $3$ conclui-se que os resultados mais fidedignos são os obtidos no ponto $4$, uma vez que os erros das grandezas obtidas para este caso são menores.
%%%%%%%%%%%%%%%%%%%%%%%%%%% Conclusões e Críticas %%%%%%%%%%%%%%%%%%%%%%%%%%%%%%
\section{Conclusões e Críticas}
\label{s:conclu}
%Incluir melhorias propostas à experiência
Na primeira parte do trabalho determinou-se a relação entre as espiras do transformador utilizado, sendo que se chegou à relação de $\frac{n_{pri}}{n_{sec}}=8,63\pm0,01$. Este valor é fiável uma vez que o resultado obtido através do ajuste linear e através da divisão do número de espiras são compatíveis.
Observou-se uma curva de histerese do núcleo do transformador de acordo com o esperado.  
No estudo em vazio do transformador conseguiram-se estados de magnetização semelhantes, para a fonte aplicada no primário e secundário. Os valores obtidos para os parâmetros do esquema equivalente de Steinmetz são plausíveis, conforme discutido na análise de resultados.
Os valores teóricos comparados com os experimentais para os ensaios com o secundário em aberto e em carga são próximos, com desvios à exactidão de cerca de 10\% à excepção da eficiência, que para um resistência de $R=10\Omega$ apresenta um desvio de 57\%. Para valores baixos da resistência de carga, os valores da resistência das ligações eléctricas (algo instáveis na montagem experimental), da mesma ordem de grandeza, podem influenciar os resultados obtidos.



%\begin{acknowledgments}
%\end{acknowledgments}

%%%%%%%%%%%%%%%%%%%%%%%%%%%%%%%%%%%%%%%%%%%%%%%%%%%%%%%%%%%%%%%%%%%%%%%%%%%%%%%%
% % % % % % % % % % % % % % % %     FIM    % % % % % % % % % % % % % % % % % % % 
%%%%%%%%%%%%%%%%%%%%%%%%%%%%%%%%%%%%%%%%%%%%%%%%%%%%%%%%%%%%%%%%%%%%%%%%%%%%%%%%

\nocite{*}
\bibliography{bibliografia}{}
\bibliographystyle{plain}% Produces the bibliography via BibTeX.
\end{document}
%end of file

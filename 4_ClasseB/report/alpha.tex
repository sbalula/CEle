% ****** Start of file .tex ******
%
%   This file is based on apssamp.tex, part of the APS files in the REVTeX 4.1 distribution.
%   Version 4.1r of REVTeX, August 2010
%
%   Copyright (c) 2009, 2010 The American Physical Society.
%   Samuel Balula, Pedro Ribeiro, Luís Macedo, Eduardo Neto 2013
%   See the REVTeX 4 README file for restrictions and more information.
%
% TeX'ing this file requires that you have AMS-LaTeX 2.0 installed
% as well as the rest of the prerequisites for REVTeX 4.1
%
% See the REVTeX 4 README file
% It also requires running BibTeX. The commands are as follows:
%
%  1)  latex filename.tex
%  2)  bibtex filename
%  3)  latex filename.tex
%  4)  latex filename.tex

\documentclass[%
  reprint,
  %superscriptaddress,
  %groupedaddress,
  %unsortedaddress,
  %runinaddress,
  %frontmatterverbose, 
  %preprint,
  %showpacs,preprintnumbers,
  nofootinbib,
  %nobibnotes,
  %bibnotes,
  amsmath,amssymb,
  aps,
  %pra,
  %prb,
  %rmp,
  %prstab,
  %prstper,
  %floatfix,
  10pt,
  a4paper
]{revtex4-1}



\usepackage{facil}                      % Pacote pessoal
\usepackage{verbatim}                   % Apresentação de código
\usepackage{graphicx}                   % Include figure files
\usepackage{dcolumn}                    % Align table columns on decimal point
\usepackage{bm}                         % bold math
\usepackage[latin1,utf8]{inputenc}      % Tipos de caracteres
\usepackage[portuges]{babel}            % Português
\usepackage{indentfirst}                % Identação da primeira linha
\usepackage{hyperref}                   % add hypertext capabilities
\usepackage{float}                      %Fixar imagens
%\usepackage[mathlines]{lineno}          % Enable numbering of text and display math
%\linenumbers\relax                      % Commence numbering lines
%\usepackage[compact]{titlesec}

\usepackage[%showframe,%Uncomment any one of the following lines to test 
%%scale=0.7, marginratio={1:1, 2:3}, ignoreall, % default settings
%%text={7in,10in},centering,
margin=0.5in,        %diminuir margens
%total={6.5in,8.75in}, top=1.2in, left=0.9in, 
includefoot
%height=10in,a5paper,hmargin={3cm,0.8in},
]{geometry}

\begin{document}
\preprint{APS/123-QED}
%\captionsetup[table]{font=small,skip=0pt}
%\captionsetup[figure]{font=small,skip=0pt}
%\titlespacing{\section}{0pt}{*0}{*0}            %Poupar espaço
%\titlespacing{\subsection}{0pt}{*0}{*0}
%\titlespacing{\subsubsection}{0pt}{*0}{*0}


% % % % % % % % % % % % % % % % % % % % % % % % % % % % % % % % % % % % % % % % 
%%%%%%%%%%%%%%%%%%%%%%%%%%%%%%%%%% Início %%%%%%%%%%%%%%%%%%%%%%%%%%%%%%%%%%%%%%
% % % % % % % % % % % % % % % % % % % % % % % % % % % % % % % % % % % % % % % %
 

\title{Andar final de amplificação em classe B\\
com transistores bipolares 2n3055}
\thanks{}

\author{Pedro Ribeiro}%
\email{73221, pedro.q.ribeiro@tecnico.ulisboa.pt}
\author{Luis Macedo}%
\email{73633, luis.macedo@tecnico.ulisboa.pt}
\author{Samuel Balula}%
\email{72735, samuel.balula@tecnico.ulisboa.pt}

\affiliation{
  Instituto Superior Técnico\\
  Mestrado em Engenharia Física Tecnológica\\
  Complementos de Electrónica
}

%\collaboration{Grupo 57}

\date{\today}

%%%%%%%%%%%%%%%%%%%%%%%%%%%%%%%%%% Abstract %%%%%%%%%%%%%%%%%%%%%%%%%%%%%%%%%%%%
\begin{abstract}
Neste trabalho laboratorial é montado um andar de saída em classe B, obtendo-se a característica de tensão e corrente, a partir das quais se determina o respectivo ganho. Determina-se a potência fornecida e dissipada por cada um dos elementos do circuito, a eficiência, as impedâncias de entrada e saída, e o ganho de corrente em função da frequência.

\end{abstract}
\maketitle


%%%%%%%%%%%%%%%%%%%%%%%%%%%%%%%%%% Introdução %%%%%%%%%%%%%%%%%%%%%%%%%%%%%%%%%%
\section{Introdução}
\label{s:intro}
%Situar o problema, incluir fórmulas.
%Quem lê o relatório deve conseguir perceber exatamente o que foi feito.
O andar final de amplificação em classe B permite fornecer o sinal a uma carga de impedância baixa, com uma eficiência superior ao andar em classe A. Neste apenas um dos transístores está em condução a qualquer momento, o que leva a uma distorção de cruzamento para tensões de entrada próximas de zero. São usados em circuitos em que a potência disponível é limitada e a distorção do sinal é admissível, embora existam técnicas para reduzir grandemente este efeito, como pela utilização de amplificadores operacionais




\subsection{Função de transferência}
Pela análise do circuito da \rfig{../img/esquematico.png}, podem escrever-se as equações \eqref{opedroefeio} e \eqref{oluisefeio}.
\eq[opedroefeio]{v_i - v_{R_2} - v_{BE} -v_o = 0}
\eq[oluisefeio]{v_i - v_{R_2} + v_{EB} - v_o = 0}

Conhecem-se também as equações características do transístor npn, \eqref{npncara1} e \eqref{npncara2}, apresentando expressões análogas para o transístor pnp.

\eq[npncara1]{i_E = I_{ES} \lr{e^{\frac{v_{BE}}{\eta V_t}} - 1}}
\eq[npncara2]{i_C = \beta_f i_B}

Assumindo Q1 em condução e Q2 em corte:
\eq{v_i = v_o \lr{\frac{1}{R_1 (\beta_f + 1)} + 1} + \eta V_t \ln{\lr{\frac{v_o}{R_1}\frac{1}{I_{ES}} + 1}}}

\eq{G_v = \frac{\partial v_0}{\partial v_i} = \lr{\frac{1}{R_1 (\beta_f + 1)} + \eta V_t \frac{
\frac{1}{I_{ES} R_1}
}{
\frac{v_o}{I_{ES} R_1} + 1} + 1}^{-1}}

\subsection{Potências dissipadas e eficiência}
O andar de saída em classe B como tem dois transístores que funcionam alternadamente para amplificar o sinal de entrada, possui uma eficiência bastante superior quando comparado com a eficiência do andar de classe A.\\
Teoricamente a potência fornecida pelas fontes de alimentação do andar de saída em que a fonte positiva fornece uma tensão igual em módulo à da fonte negativa é dada por:
\begin{equation}
P_{forn}=V_{CC}<i_{C1}>+V_{CC}<i_{C1}>
\label{eq:p_fonte}
\end{equation}
Como as fonte de tensão positiva só fornece tensão à parte positiva do sinal de entrada e a fonte negativa apenas à parte negativa, pode-se afirmar que $<i_{C1}>$ é o valor médio de uma sinusoide simplesmente rectificada e logo a média é calculada como (desprezando a distorção de cruzamento):
\begin{equation}
<i_{C1}>=\frac{1}{2\pi}\int_0^{\pi} \! I_{o_m}\sin(\theta) \, \mathrm{d}\theta=\frac{V_{o_m}}{R_L\pi}
\label{eq:curr}
\end{equation}
Visto que se está a desprezar a distorção de cruzamento e que o sinal é sinusoidal, pode-se considerar que $<i_{C1}>=<i_{C2}>$ e introduzindo a corrente determinada em \ref{eq:curr} na equação para a potência das fontes \ref{eq:p_fonte}, obtém-se:
\begin{equation}
P_{forn}=2\frac{V_{CC}V_{o_m}}{R_L\pi}
\label{eq:p_fonte_f}
\end{equation}
Finalmente, visto que a potência fornecida à carga é dada por:
\begin{equation}
P_L=\frac{V_{o_m}^2}{2R_L}
\end{equation}
A eficiência é dada por:
\begin{equation}
\eta=\frac{P_L}{P_{forn}}=\frac{\pi V_{o_m}}{4V_{CC}}
\label{eq:eff}
\end{equation}
Por análise da equação \ref{eq:eff}, nota-se que a eficiência é máxima quando a tensão de saída é máxima. O valor de tensão máximo é dado $V_{o_m}\approx V_{CC}$ se se desprezar a tensão de saturação do transístor e consequentemente a eficiência máxima é dada por:
\begin{equation}
\eta_{\mathrm{max}}\approx \frac{\pi}{4}=78.5\%
\end{equation}
Finalmente, a potência dissipada pelos transístores pode ser calculada fazendo:
\begin{equation}
P_D=P_{forn}-P_L=\frac{2V_{CC}V_{o_m}}{\pi R_L}-\frac{V_{o_m}^2}{2R_L}
\label{eq:p_diss}
\end{equation}

%%%%%%%%%%%%%%%%%%%%%%%%%%%% Experiência realizada %%%%%%%%%%%%%%%%%%%%%%%%%%%%%
\section{Experiência Realizada}
\label{s:expreal}
%Incluir diagrama de blocos da montagem
De acordo com as instruções do guia, implentou-se o circuito da \rfig{../img/esquematico.png}. Apresentam-se na tabela \ref{partlist} os componentes utilizados.

\fig{../img/esquematico.png}{Circuito implementado no laboratório}

\tabela[partlist]{Lista dos componentes utilizados}{lrrr}{
	Descrição		&Modelo/Valor		&Qt.			&Referência	\\ \hline
	TJB NPN			&2N3055			&1			&Q1	\\
	TJB PNP			&MJ2955			&1			&Q2		\\
	Resistência 		&10$\Omega$		&2			&R1, R2		\\
}


%%%%%%%%%%%%%%%%%%%%%%%%%%%%%%%% Resultados %%%%%%%%%%%%%%%%%%%%%%%%%%%%%%%%%%%%
\section{Resultados}
\label{s:resul}
%incluir tabelas. Excesso de dados => grafico com valores mais significativos

Tal como sugerido no guia, verificou-se que anulando a entrada ambos os transístores estão em corte, a tensão de saída anula-se e não há dissipação de potência no andar de saída

\subsection{Função de transferência}

\fig{../gnuplot/vo(vi).pdf}{Função de transferência vo(vi) para tensão alternada sinusoidal de 300$Hz$. Ajuste pelo método dos mínimos quadrados a uma função do tipo $y=mx+b$. Parâmetros de ajuste: $m=0.978\pm0.015$, $b=-0.96\pm0.07 V$, $Chi_{ndf}=0.01$}

\fig{../gnuplot/io(ii).pdf}{Característica de corrente io(ii) para tensão alternada sinusoidal de 300$Hz$.
Ajuste pelo método dos mínimos quadrados a uma função do tipo $y=mx+b$. Parâmetros de ajuste: $m=118\pm1$, $b=-0.079\pm0.005 (A)$, $Chi_{ndf}=7.3\e{-5}$}

\fig{../gnuplot/xy.pdf}{Dados experimentais obtidos no osciloscópio no modo xy para a característica vo(vi)}

%\fig{../ngspice/plot.pdf}{Função de transferência vo(vi) obtida na simulação em ngspice}

Apresenta-se na \rfig{../gnuplot/vo(vi).pdf} a funçao de transferência vo(vi) e na \rfig{../gnuplot/io(ii).pdf} a característica de corrente io(ii). Procedeu-se a um ajuste aos pontos (sem considerar erros experimentais), com o qual se determinou o ganho de tensão $0.978\pm0.015$ e de corrente $118\pm1$. Apresenta-se também na \rfig{../gnuplot/xy.pdf} uma aquisição do osciloscópio no modo xy.
Determinou-se que o valor máximo de vo para o qual não há distorção é $6.57 V rms$, isto é $9.29 V$ de amplitude.

A distorção ocorre quando a tensão de saída atinge $V_{dd} - V_{ce_{saturacao}}$ ou $V_{ss} + V_{ec_{saturacao}}$, sendo que neste caso os valores são iguais em módulo. Como $V_{ce_{saturacao}}~0.7V$, $vo_{max}\approx9.7 V$.


\subsection{Potências fornecidas e dissipadas}



\subsection{Impedâncias de entrada e saída}
\subsection{Resposta em frequência}






%%%%%%%%%%%%%%%%%%%%%%%%%%% Análise dos resultados %%%%%%%%%%%%%%%%%%%%%%%%%%%%%
\section{Análise de resultados}
\label{s:aresul}

\subsection{Função de transferência}

Os valores obtidos para o ganho de tensão e de corrente estão de acordo com o esperado, com 

O valor obtido para a tensão máxima sem distorção está de acordo com o obtido experimentalmente.

\subsection{Potências fornecidas e dissipadas}
\subsection{Impedâncias de entrada e saída}
\subsection{Resposta em frequência}


%%%%%%%%%%%%%%%%%%%%%%%%%%% Conclusões e Críticas %%%%%%%%%%%%%%%%%%%%%%%%%%%%%%
\section{Conclusões e Críticas}
\label{s:conclu}
%Incluir melhorias propostas à experiência


%\begin{acknowledgments}
%\end{acknowledgments}

%%%%%%%%%%%%%%%%%%%%%%%%%%%%%%%%%%%%%%%%%%%%%%%%%%%%%%%%%%%%%%%%%%%%%%%%%%%%%%%%
% % % % % % % % % % % % % % % %     FIM    % % % % % % % % % % % % % % % % % % % 
%%%%%%%%%%%%%%%%%%%%%%%%%%%%%%%%%%%%%%%%%%%%%%%%%%%%%%%%%%%%%%%%%%%%%%%%%%%%%%%%

\nocite{*}
\bibliography{bibliografia}{}
\bibliographystyle{plain}% Produces the bibliography via BibTeX.
\end{document}
%end of file

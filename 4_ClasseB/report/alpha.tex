% ****** Start of file .tex ******
%
%   This file is based on apssamp.tex, part of the APS files in the REVTeX 4.1 distribution.
%   Version 4.1r of REVTeX, August 2010
%
%   Copyright (c) 2009, 2010 The American Physical Society.
%   Samuel Balula, Pedro Ribeiro, Luís Macedo, Eduardo Neto 2013
%   See the REVTeX 4 README file for restrictions and more information.
%
% TeX'ing this file requires that you have AMS-LaTeX 2.0 installed
% as well as the rest of the prerequisites for REVTeX 4.1
%
% See the REVTeX 4 README file
% It also requires running BibTeX. The commands are as follows:
%
%  1)  latex filename.tex
%  2)  bibtex filename
%  3)  latex filename.tex
%  4)  latex filename.tex

\documentclass[%
  reprint,
  %superscriptaddress,
  %groupedaddress,
  %unsortedaddress,
  %runinaddress,
  %frontmatterverbose, 
  %preprint,
  %showpacs,preprintnumbers,
  nofootinbib,
  %nobibnotes,
  %bibnotes,
  amsmath,amssymb,
  aps,
  %pra,
  %prb,
  %rmp,
  %prstab,
  %prstper,
  %floatfix,
  10pt,
  a4paper
]{revtex4-1}



\usepackage{facil}                      % Pacote pessoal
\usepackage{verbatim}                   % Apresentação de código
\usepackage{graphicx}                   % Include figure files
\usepackage{dcolumn}                    % Align table columns on decimal point
\usepackage{bm}                         % bold math
\usepackage[latin1,utf8]{inputenc}      % Tipos de caracteres
\usepackage[portuges]{babel}            % Português
\usepackage{indentfirst}                % Identação da primeira linha
\usepackage{hyperref}                   % add hypertext capabilities
\usepackage{float}                      %Fixar imagens
%\usepackage[mathlines]{lineno}          % Enable numbering of text and display math
%\linenumbers\relax                      % Commence numbering lines
%\usepackage[compact]{titlesec}

\usepackage[%showframe,%Uncomment any one of the following lines to test 
%%scale=0.7, marginratio={1:1, 2:3}, ignoreall, % default settings
%%text={7in,10in},centering,
margin=0.5in,        %diminuir margens
%total={6.5in,8.75in}, top=1.2in, left=0.9in, 
includefoot
%height=10in,a5paper,hmargin={3cm,0.8in},
]{geometry}


\if11
%%%%%%%%%%%%%%% Poupar espaço
%\oddsidemargin = 0pt                    % Margem do lado esquerdo: 31pt
\topmargin = -20mm                        % Margem superior: 20pt  
\headheight = 10pt                       % Tamanho do 'header': 12pt 
\headsep = 3mm                      % Espaço entre o 'header' e o texto: 25pt
\textheight = 275mm                      % Altura do texto: 592pt
\textwidth = 190mm                      % Largura do texto: 390pt
%\marginparsep = 10pt                     % Espaço entre margem esquerda e o texto: 10pt
%\marginparwidth = 35pt                   % Margem esquerda: 35pt
\footskip = 0pt                         % Espaço entre o texto e o 'footer': 30pt
\oddsidemargin = -15mm
\abovedisplayskip = 1mm
\belowdisplayskip = 1mm
\textfloatsep = 1mm

\usepackage[compact]{titlesec}
%%%%%%%%%%%%%
\fi


\begin{document}
\preprint{APS/123-QED}


% % % % % % % % % % % % % % % % % % % % % % % % % % % % % % % % % % % % % % % % 
%%%%%%%%%%%%%%%%%%%%%%%%%%%%%%%%%% Início %%%%%%%%%%%%%%%%%%%%%%%%%%%%%%%%%%%%%%
% % % % % % % % % % % % % % % % % % % % % % % % % % % % % % % % % % % % % % % %
 

\title{Andar final de amplificação em classe B\\
com transistores bipolares 2n3055}
%\thanks{}

\author{Pedro Ribeiro}%
\email{73221, pedro.q.ribeiro@tecnico.ulisboa.pt}
\author{Luis Macedo}%
\email{73633, luis.macedo@tecnico.ulisboa.pt}
\author{Samuel Balula}%
\email{72735, samuel.balula@tecnico.ulisboa.pt}

\affiliation{
  Instituto Superior Técnico\\
  Mestrado em Engenharia Física Tecnológica\\
  Complementos de Electrónica
}

%\collaboration{Grupo 57}

\date{\today}

%%%%%%%%%%%%%%%%%%%%%%%%%%%%%%%%%% Abstract %%%%%%%%%%%%%%%%%%%%%%%%%%%%%%%%%%%%
\begin{abstract}
Neste trabalho laboratorial é montado um andar de saída em classe B, obtendo-se a característica de tensão e corrente, a partir das quais se determina um ganho de tensão $0.978\pm0.015$ e de corrente $118\pm1$. Determina-se a potência fornecida e dissipada por cada um dos elementos do circuito, a eficiência, as impedâncias de entrada e saída, e o ganho de corrente em função da frequência, tendo-se obtido uma eficiência de 68.3\% e uma potência dissipada de 2W.  As impedâncias obtidas para o circuito foram de $Z_{in}=INTRODUZIR VALOR!!!!!!!$, $Z_{o1}=2.43 \pm 0.06 \Omega$ e $Z_{o2}=0.90 \pm 0.03 \Omega$. Concluiu-se ainda que o limite superior da banda passante a $-3dB$ se encontra em $63kHz$.

\end{abstract}
\maketitle


%%%%%%%%%%%%%%%%%%%%%%%%%%%%%%%%%% Introdução %%%%%%%%%%%%%%%%%%%%%%%%%%%%%%%%%%
\section{Introdução}
\label{s:intro}
%Situar o problema, incluir fórmulas.
%Quem lê o relatório deve conseguir perceber exatamente o que foi feito.
O andar final de amplificação em classe B permite fornecer o sinal a uma carga de impedância baixa, com uma eficiência superior ao andar em classe A. Neste apenas um dos transístores está em condução a qualquer momento, o que leva a uma distorção de cruzamento para tensões de entrada próximas de zero. São usados em circuitos em que a potência disponível é limitada e a distorção do sinal é admissível, embora existam técnicas para reduzir grandemente este efeito, como pela utilização de amplificadores operacionais




\subsection{Função de transferência}
Pela análise do circuito da \rfig{../img/esquematico.png}, podem escrever-se as equações \eqref{opedroefeio} e \eqref{oluisefeio}.
\eq[opedroefeio]{v_i - v_{R_2} - v_{BE} -v_o = 0}
\eq[oluisefeio]{v_i - v_{R_2} + v_{EB} - v_o = 0}

Conhecem-se também as equações características do transístor npn, \eqref{npncara1} e \eqref{npncara2}, apresentando expressões análogas para o transístor pnp.

\eq[npncara1]{i_E = I_{ES} \lr{e^{\frac{v_{BE}}{\eta V_t}} - 1}}
\eq[npncara2]{i_C = \beta_f i_B}

Assumindo Q1 em condução e Q2 em corte:
\eq{v_i = v_o \lr{\frac{1}{R_1 (\beta_f + 1)} + 1} + \eta V_t \ln{\lr{\frac{v_o}{R_1}\frac{1}{I_{ES}} + 1}}}

\eq{G_v = \frac{\partial v_0}{\partial v_i} = \lr{\frac{1}{R_1 (\beta_f + 1)} + \eta V_t \frac{
\frac{1}{I_{ES} R_1}
}{
\frac{v_o}{I_{ES} R_1} + 1} + 1}^{-1}}

Obtêm-se uma expressão equivalente para Q2 em condução e Q1 em corte. Para os valores em causa, $G_v \sim 1$. Para a corrente, pode escrever-se \eqref{transi}, sendo o ganho $G_i = \beta_f +1$.
\eq[transi]{i_o = (\beta_f + 1) i_i}

\subsection{Potências dissipadas e eficiência}
O andar de saída em classe B como tem dois transístores que funcionam alternadamente para amplificar o sinal de entrada, possui uma eficiência bastante superior quando comparado com a eficiência do andar de classe A.\\
Teoricamente a potência fornecida pelas fontes de alimentação do andar de saída em que a fonte positiva fornece uma tensão igual em módulo à da fonte negativa é dada por:
\begin{equation}
P_{forn}=V_{CC}<i_{C1}>+V_{CC}<i_{C1}>
\label{eq:p_fonte}
\end{equation}
Como as fonte de tensão positiva só fornece tensão à parte positiva do sinal de entrada e a fonte negativa apenas à parte negativa, pode-se afirmar que $<i_{C1}>$ é o valor médio de uma sinusoide simplesmente rectificada e logo a média é calculada como (desprezando a distorção de cruzamento):
\begin{equation}
<i_{C1}>=\frac{1}{2\pi}\int_0^{\pi} \! I_{o_m}\sin(\theta) \, \mathrm{d}\theta=\frac{V_{o_m}}{R_L\pi}
\label{eq:curr}
\end{equation}
Visto que se está a desprezar a distorção de cruzamento e que o sinal é sinusoidal, pode-se considerar que $<i_{C1}>=<i_{C2}>$ e introduzindo a corrente determinada em \ref{eq:curr} na equação para a potência das fontes \ref{eq:p_fonte}, obtém-se:
\begin{equation}
P_{forn}=2\frac{V_{CC}V_{o_m}}{R_L\pi}
\label{eq:p_fonte_f}
\end{equation}
Finalmente, visto que a potência fornecida à carga é dada por:
\begin{equation}
P_L=\frac{V_{o_m}^2}{2R_L}
\end{equation}
A eficiência é dada por:
\begin{equation}
\eta=\frac{P_L}{P_{forn}}=\frac{\pi V_{o_m}}{4V_{CC}}
\label{eq:eff}
\end{equation}
Por análise da equação \ref{eq:eff}, nota-se que a eficiência é máxima quando a tensão de saída é máxima. O valor de tensão máximo é dado $V_{o_m}\approx V_{CC}$ se se desprezar a tensão de saturação do transístor e consequentemente a eficiência máxima é dada por:
\begin{equation}
\eta_{\mathrm{max}}\approx \frac{\pi}{4}=78.5\%
\end{equation}
Finalmente, a potência dissipada pelos transístores pode ser calculada fazendo:
\begin{equation}
P_D=P_{forn}-P_L=\frac{2V_{CC}V_{o_m}}{\pi R_L}-\frac{V_{o_m}^2}{2R_L}
\label{eq:p_diss}
\end{equation}
Atráves da equação \ref{eq:p_diss}, tem-se para os valores implementados no laboratório (que podem ser consultados na tabela \ref{tab:fixe}) que a potência dissipada pelo andar é de:
\begin{equation}
P_D\approx1.35 W
\end{equation}
Para calcular a eficiência experimentalmente é necessário o cálculo das potências fornecidas por ambas as fontes e a potência consumida pela resistência de carga. Visto que o andar é puramente resistivo, o valor das potência fornecida pela fonte $i$ é calculado por:
\begin{equation}
P_{forn_i}=V_i \cdot I_i(\mathrm{rms})
\end{equation}
A potência consumida pela resistência de carga é dada por:
\begin{equation}
P_L=\frac{(V_L(\mathrm{rms}))^2}{R_L}
\end{equation}
Sendo que finalmente a eficiência é dada do andar é dada por:
\begin{equation}
\eta=\frac{P_L}{P_{forn_i}}=\frac{V_i \cdot I_i(\mathrm{rms})}{\frac{(V_L(\mathrm{rms}))^2}{R_L}}
\end{equation}

\subsection{Impedâncias de entrada e saída}
Com a finalidade de determinar a impedância de entrada do circuito mede-se a corrente e a tensão de entrada utilizando um multímetro e determina-se a impedância de entrada a partir da equação \ref{eq:impedanciaentrada}.
\begin{equation}
Z_{in}=\frac{V_{in}}{I_{in}}
\label{eq:impedanciaentrada}
\end{equation}
Para determinar a impedância de saída utilizam-se duas cargas diferentes e mede-se a queda de tensão nelas. Considera-se que todo o circuito é linear e portanto pode ser representado por um equivalente de Thévenin (uma fonte de tensão e uma impedância em série, neste caso essa impedância corresponde à impedância de saída do circuito). A impedância de saída é então dada pela equação \ref{eq:impedanciasaida}

\begin{equation}
Z_{out}=\frac{R_1 \frac{dV_0}{dR_1}}{\frac{V_0}{R_1}-\frac{dV_0}{dR_1}}
\label{eq:impedanciasaida}
\end{equation}

\subsection{Resposta em frequência}
Pretende-se verificar a resposta em frequência do circuito e para tal varria-se a frequência do sinal de entrada e apresenta-se o gráfico de $\frac{i_o}{i_i} (dB)$ em função de $log(f)$. Para determinar as correntes de entrada e de saída medem-se as quedas de tensão nas resistências à entrada e à saída do circuito e utilizando $I=V/R$ obtêm-se as correntes. Determina-se ainda o limite superior da banda passante a $-3 dB$ a partir da intersecção dessa recta com o gráfico já referido.
%%%%%%%%%%%%%%%%%%%%%%%%%%%% Experiência realizada %%%%%%%%%%%%%%%%%%%%%%%%%%%%%
\section{Experiência Realizada}
\label{s:expreal}
%Incluir diagrama de blocos da montagem
De acordo com as instruções do guia, implentou-se o circuito da \rfig{../img/esquematico.png}. Apresentam-se na tabela \ref{partlist} os componentes utilizados.

\fig{../img/esquematico.png}{Circuito implementado no laboratório}

\tabela[partlist]{Lista dos componentes utilizados}{lrrr}{
	Descrição		&Modelo/Valor		&Qt.			&Referência	\\ \hline
	TJB NPN			&2N3055			&1			&Q1	\\
	TJB PNP			&MJ2955			&1			&Q2		\\
	Resistência 		&10$\Omega$		&2			&R1, R2		\\
}


%%%%%%%%%%%%%%%%%%%%%%%%%%%%%%%% Resultados %%%%%%%%%%%%%%%%%%%%%%%%%%%%%%%%%%%%
\section{Resultados}
\label{s:resul}
%incluir tabelas. Excesso de dados => grafico com valores mais significativos

Tal como sugerido no guia, verificou-se que anulando a entrada ambos os transístores estão em corte, a tensão de saída anula-se e não há dissipação de potência no andar de saída

\subsection{Função de transferência}

\fig{../gnuplot/vo(vi).pdf}{Função de transferência vo(vi) para tensão alternada sinusoidal de 300$Hz$. Ajuste pelo método dos mínimos quadrados a uma função do tipo $y=mx+b$. Parâmetros de ajuste: $m=0.978\pm0.015$, $b=-0.96\pm0.07 V$, $Chi_{ndf}=0.01$}

\fig{../gnuplot/io(ii).pdf}{Característica de corrente io(ii) para tensão alternada sinusoidal de 300$Hz$.
Ajuste pelo método dos mínimos quadrados a uma função do tipo $y=mx+b$. Parâmetros de ajuste: $m=118\pm1$, $b=-0.079\pm0.005 (A)$, $Chi_{ndf}=7.3\e{-5}$}

\fig{../gnuplot/xy.pdf}{Dados experimentais obtidos no osciloscópio no modo xy para a característica vo(vi)}

%\fig{../ngspice/plot.pdf}{Função de transferência vo(vi) obtida na simulação em ngspice}

Apresenta-se na \rfig{../gnuplot/vo(vi).pdf} a funçao de transferência vo(vi) e na \rfig{../gnuplot/io(ii).pdf} a característica de corrente io(ii). Procedeu-se a um ajuste aos pontos (sem considerar erros experimentais), com o qual se determinou o ganho de tensão $0.978\pm0.015$ e de corrente $118\pm1$. Apresenta-se também na \rfig{../gnuplot/xy.pdf} uma aquisição do osciloscópio no modo xy.
Determinou-se que o valor máximo de vo para o qual não há distorção é $6.57 V rms$, isto é $9.29 V$ de amplitude.

A distorção ocorre quando a tensão de saída atinge $V_{dd} - V_{ce_{saturacao}}$ ou $V_{ss} + V_{ec_{saturacao}}$, sendo que neste caso os valores são iguais em módulo. Como $V_{ce_{saturacao}}\sim0.5V$, $vo_{max}\approx9.5 V$.


\subsection{Potências fornecidas e dissipadas}
Os resultados obtidos para as medições quando $v_0=v_{0_m}$ necessárias para calcular as potências dissipadas e o rendimento, juntamente com estas quantidades estão apresentados na tabela \ref{tab:fixe}.
\begin{table}[h]
\begin{tabular}{c|c}
                  & $v_{o}=v_{o_m}$ \\ \hline
$V_{+}$ (V)       & 9.26            \\ \hline
$V_{-}$ (V)       & -9.89           \\ \hline
$I_{+}$ (rms) (A) & 0.33            \\ \hline
$I_{-}$ (rms) (A) & 0.33            \\ \hline
$v_{0}$ (rms) (V) & 6.57            \\ \hline
$P_{R_L}$ (W)     & 4.32            \\ \hline
$P_{forn_+}$ (W)  & 3.05            \\ \hline
$P_{forn_-}$ (W)  & 3.26           \\ \hline
$P_{Q_N}$ (W)     & 0.896           \\ \hline
$P_{Q_P}$ (W)     & 1.104           \\ \hline
Eficiência (\%)   & 68.3          
\end{tabular}
\caption{Resultados obtidos o rendimento e valores necesários para o cálculo deste}
\label{tab:fixe}
\end{table}
O rendimento obtido foi de 68.3\%, ligeiramente mais baixo que o rendimento previsto de 78.5\%
Considerou-se para o cálculo das potências dissipadas considerou-se que metade da potência dissipada pela resistência de carga é fornecida pela fonte positiva, e a outra metade pela fonte negativa.

\subsection{Impedâncias de entrada e saída}

Para determinar a impedância de entrada mediu-se directamente, utilizando um multimetro, a tensão de entrada e mediu-se a tensão aos terminais de uma resistência de $10 \Omega$ de modo a determinar a corrente de entrada. Mediram-se também, utilizando duas cargas diferentes, a corrente e tensão nessas cargas, de modo a que, a partir do equivalente de Thévenin, fosse possível determinar a impedância de saída do circuito. Os resultados estão na tabela\ref{tab:impedanciadados}.

\begin{table}[h]
    \begin{tabular}{|l|l|}
    \hline
    $V_i $(V) & ~ \\ \hline
    $I_i$ (A) & ~ \\ \hline
    R1($\Omega$)    & $10.0\pm 0.1$ \\ \hline
    $V_{o1}$ (V)   & $0.893 \pm 0.001$ \\ \hline
    R2 ($\Omega$)   & $6.1\pm 0.1$ \\ \hline
    $V_{o2}$ (V)   & $0.825\pm 0.001$ \\ \hline
    \end{tabular}
\caption{Dados obtidos para a determinação das impedâncias de entrada e de saída}
\label{tab:impedanciadados}
\end{table}



\subsection{Resposta em frequência}
Variando a frequência do sinal de entrada mediram-se as tensões de entrada e de saída. Os resultados encontram-se na tabela \ref{tab:respostafrequencia}.
\begin{table}[h]
    \begin{tabular}{|l|l|l|l|l|}
    \hline
    $F_{sinal de entrada} (Hz)$ & Vi(V) & Vo(V)& Log(f) (Hz)& $i_o/i_i (dB)$ \\ \hline
    100                                               & 0.88  & 7.76 & 2.0	& 16.6                                  \\ \hline
    200                              		    & 0.88  & 7.76 & 2.3	&16.6		\\ \hline
    400                              		    & 0.80 & 7.92 & 2.6	&17.6			\\ \hline
    600                              		    & 0.72  & 8.32 & 2.8	&19.0			\\ \hline
    800                              		    & 0.72  & 8.42 & 2.9	&19.1			\\ \hline
    1000                             		    & 0.72  & 8.64&  3.0	&19.3			\\ \hline
    2000                             		    & 0.56  & 8.96&  3.3	&21.8			\\ \hline
    4000                             		    & 0.56  & 9.20&  3.6	&22.0			\\ \hline
    6000                             		    & 0.56  & 9.36&  3.8	&22.2			\\ \hline
    8000                             		    & 0.40  & 9.44&  3.9	&25.2			\\ \hline
    10000                            		    & 0.40  & 9.44&  4.0	&25.2				\\ \hline
    20000                            		    & 0.40  & 9.44&  4.3	&25.2				\\ \hline
    40000                            		    & 0.32  & 9.36&  4.6	&27.1				\\ \hline
    60000                            		    & 0.48  & 9.36&  4.8	&23.5				\\ \hline
    80000                            		    & 0.48  & 9.36&  4.9	&23.5				\\ \hline
    100000                           	    & 0.64  & 9.44&  5.0	&21.1				\\ \hline
    200000                           	    & 1.04  & 9.36&  5.3	&16.8			\\ \hline
    400000                           	    & 1.36  & 9.44&  5.6	&14.6				\\ \hline
    600000                           	    & 1.52  & 9.04&  5.8	&13.2				\\ \hline
    800000                           	    & 1.92  & 9.12&  5.9	&11.3				\\ \hline
    1000000                          	    & 2.24  & 8.72&  6.0	&9.5				\\ \hline
    2000000				    &4.4     &8.16 &  6.3     &3.1                          \\ \hline
    \end{tabular}
\caption{Dados obtidos para a determinação da resposta em frequência do circuito}
\label{tab:respostafrequencia}
\end{table}






%%%%%%%%%%%%%%%%%%%%%%%%%%% Análise dos resultados %%%%%%%%%%%%%%%%%%%%%%%%%%%%%
\section{Análise de resultados}
\label{s:aresul}

\subsection{Função de transferência}

Os valores obtidos para o ganho de tensão e de corrente estão de acordo com o esperado, obtendo-se $\beta_f = 117\pm1$, o que é coerente com o indicado na figura 3 de \cite{data}, para uma temperatura ambiente de cerca de 25 ºC e uma corrente de cerca de 1A.

O valor obtido para a tensão máxima sem distorção está aproximadamente de acordo com o obtido experimentalmente.

Ambas as funções de transferência apresentam um comportamento linear.

\subsection{Potências fornecidas e dissipadas}
Dos resultados da tabela \ref{tab:fixe}, conclui-se que a eficiência do andar estudado é de 68.3\% quando a tensão de entrada é a máxima sem a existência de distorção. Este valor é ligeiramente inferior ao valor previsto pela teoria de 78.5\%, diferença essa que pode ser explicada pelo facto de apenas se considerar no modelo teórico as perdas nos transístores, enquanto que nos cálculos para a eficiência experimental se contabilizam todas as perdas.\\
É interessante comentar que este andar é muito mais eficiente que o andar de saída em classe A, sendo que tanto teoricamente como experimentalmente a eficiência é 3 a 4 vezes no andar de classe B face ao andar de classe A.
Somando as potências dissipadas obtidas experimentalmente para ambas os transístores, tem-se:
\begin{equation}
P_{D_{exp}}\approx 2
\end{equation}
Valor este que é aproximadamente o dobro do valor obtido teoricamente. Mais isto pode dever-se ao facto de na formulação teórica, apenas se considerar a dissipação em transístores ideais, enquanto que na realidade, os transístores possuem muitas não-idealidades que não são contabilizadas na formulação teórica mas que são suficientemente importantes para que no final, o resultado obtido seja diferente do esperado.

\subsection{Impedâncias de entrada e saída}

A impedância de entrada foi determinada através dos dados da tabela \ref{tab:impedanciadados} e da equação \ref{eq:impedanciaentrada}. Para a impedância de saída obtiveram-se 2 valores, dado que se utilizaram duas cargas para os cálculos, utilizando a equação \ref{eq:impedanciasaida}. Considerou-se que $\frac{dV_o}{dR_1}=\frac{V_{o1}-V_{o2}}{R_1-R_2}$ quando se utilizou $R_1$ epara os cálculos e  $\frac{dV_o}{dR_1}=\frac{V_{o2}-V_{o1}}{R_2-R_1}$ quando se utilizou $R_2$. Os resultados obtidos encontram-se na tabela \ref{tab:impedanciaresultados}.

\begin{table}[h]
    \begin{tabular}{|l|l|}
    \hline
    $Z_{in} (\Omega)$ & ~ \\ \hline
    $Z_{o1} (\Omega)$ & $2.43 \pm0.06$ \\ \hline
    $Z_{o2} (\Omega)$ & $0.90 \pm 0.03$ \\ \hline
    \end{tabular}
\caption{Resultados obtidos para as impedâncias de entrada e de saída}
\label{tab:impedanciaresultados}
\end{table}

Como seria de esperar, o andar de saída em classe B tem uma impedância de entrada alta e uma impedância de saída baixa.



\subsection{Resposta em frequência}
Tendo em conta os dados da tabela \ref{tab:respostafrequencia} e que $R_i=47.0 \pm 0.1 \Omega$ e que $R_o=6.1 \pm 0.1 \Omega$ apresentam-se também na tabela \ref{tab:respostafrequencia} $Log(f)$ e $i_o / i_i $ em $dB$. Encontra-se na  \rfig{../img/Ponto_D.png} o gráfico de $i_o/i_i (dB)$ em função de $Log(f) (Hz)$.

\fig[.28]{../img/Ponto_D.png}{Resposta em frequência do circuito}
Tendo em conta que o máximo de  $i_o/i_i (dB)$ é de $27.1 dB$ traçou-se uma recta a $24.1 dB$ e verificou-se assim que o limite superior da banda passante se encontra em $63kHz$. Conclui-se assim que a atenuação a partir de $63kHz$ o ganho de corrente decresce bastante.



%%%%%%%%%%%%%%%%%%%%%%%%%%% Conclusões e Críticas %%%%%%%%%%%%%%%%%%%%%%%%%%%%%%
\section{Conclusões e Críticas}
\label{s:conclu}
%Incluir melhorias propostas à experiência
O andar de saída em classe B proporciona um ganho de tensão próximo da unidade e um ganho de corrente na ordem da centena, para as condições de trabalho utilizadas, o que está de acordo com as previsões teóricas e a caracterização do fabricante.

No que diz respeito às impedâncias do circuito verifica-se, pelos dados da tabela \ref{tab:impedanciaresultados}, que o andar de saída em classe B tem uma alta impedância de entrada e uma baixa impedância de saída, o que está de acordo com as previsões teóricas. 

Relativamente à resposta em frequência concluiu-se que o limite superior da banda passante a $-3dB$ se encontra em $63kHz$, pelo que a partir desta frequência o ganho de corrente decresce consideravelmente.

A eficiência obtida experimentalmente encontra-se dentro do valor previsto para esta montagem. Quanto à potência dissipada, observa-se que o valor experimental é bastante superior (aproximadamente o dobro) da previsão teórica, o que se pode dever à dissipação por outros elementos do circuito que não sejam os transístores. Uma maneira de eliminar esta potência dissipada seria o teste de um circuito deste tipo mas num PCB, onde as perdas resistivas nos condutores seriam muito menores.

%\begin{acknowledgments}
%\end{acknowledgments}

%%%%%%%%%%%%%%%%%%%%%%%%%%%%%%%%%%%%%%%%%%%%%%%%%%%%%%%%%%%%%%%%%%%%%%%%%%%%%%%%
% % % % % % % % % % % % % % % %     FIM    % % % % % % % % % % % % % % % % % % % 
%%%%%%%%%%%%%%%%%%%%%%%%%%%%%%%%%%%%%%%%%%%%%%%%%%%%%%%%%%%%%%%%%%%%%%%%%%%%%%%%

\nocite{*}
\bibliography{bibliografia}{}
\bibliographystyle{plain}% Produces the bibliography via BibTeX.
\end{document}
%end of file

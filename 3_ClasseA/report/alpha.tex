% ****** Start of file .tex ******
%
%   This file is based on apssamp.tex, part of the APS files in the REVTeX 4.1 distribution.
%   Version 4.1r of REVTeX, August 2010
%
%   Copyright (c) 2009, 2010 The American Physical Society.
%   Samuel Balula, Pedro Ribeiro, Luís Macedo, Eduardo Neto 2013
%   See the REVTeX 4 README file for restrictions and more information.
%
% TeX'ing this file requires that you have AMS-LaTeX 2.0 installed
% as well as the rest of the prerequisites for REVTeX 4.1
%
% See the REVTeX 4 README file
% It also requires running BibTeX. The commands are as follows:
%
%  1)  latex filename.tex
%  2)  bibtex filename
%  3)  latex filename.tex
%  4)  latex filename.tex

\documentclass[%
  reprint,
  %superscriptaddress,
  %groupedaddress,
  %unsortedaddress,
  %runinaddress,
  %frontmatterverbose, 
  %preprint,
  %showpacs,preprintnumbers,
  nofootinbib,
  %nobibnotes,
  %bibnotes,
  amsmath,amssymb,
  aps,
  %pra,
  %prb,
  %rmp,
  %prstab,
  %prstper,
  %floatfix,
  10pt,
  a4paper
]{revtex4-1}



\usepackage{facil}                      % Pacote pessoal
\usepackage{verbatim}                   % Apresentação de código
\usepackage{graphicx}                   % Include figure files
\usepackage{dcolumn}                    % Align table columns on decimal point
\usepackage{bm}                         % bold math
\usepackage[latin1,utf8]{inputenc}      % Tipos de caracteres
\usepackage[portuges]{babel}            % Português
\usepackage{indentfirst}                % Identação da primeira linha
\usepackage{hyperref}                   % add hypertext capabilities
\usepackage{float}                      %Fixar imagens
%\usepackage[mathlines]{lineno}          % Enable numbering of text and display math
%\linenumbers\relax                      % Commence numbering lines
%\usepackage[compact]{titlesec}

\usepackage[%showframe,%Uncomment any one of the following lines to test 
%%scale=0.7, marginratio={1:1, 2:3}, ignoreall, % default settings
%%text={7in,10in},centering,
margin=0.5in,        %diminuir margens
%total={6.5in,8.75in}, top=1.2in, left=0.9in, 
includefoot
%height=10in,a5paper,hmargin={3cm,0.8in},
]{geometry}

\begin{document}
\preprint{APS/123-QED}
%\captionsetup[table]{font=small,skip=0pt}
%\captionsetup[figure]{font=small,skip=0pt}
%\titlespacing{\section}{0pt}{*0}{*0}            %Poupar espaço
%\titlespacing{\subsection}{0pt}{*0}{*0}
%\titlespacing{\subsubsection}{0pt}{*0}{*0}


% % % % % % % % % % % % % % % % % % % % % % % % % % % % % % % % % % % % % % % % 
%%%%%%%%%%%%%%%%%%%%%%%%%%%%%%%%%% Início %%%%%%%%%%%%%%%%%%%%%%%%%%%%%%%%%%%%%%
% % % % % % % % % % % % % % % % % % % % % % % % % % % % % % % % % % % % % % % %
 

\title{Andar final de amplificação em classe A\\
com transistores bipolares 2n3055}
\thanks{}

\author{Pedro Ribeiro}%
\email{73221, pedro.q.ribeiro@tecnico.ulisboa.pt}
\author{Luis Macedo}%
\email{73633, luis.macedo@tecnico.ulisboa.pt}
\author{Samuel Balula}%
\email{72735, samuel.balula@tecnico.ulisboa.pt}

\affiliation{
  Instituto Superior Técnico\\
  Mestrado em Engenharia Física Tecnológica\\
  Complementos de Electrónica
}

%\collaboration{Grupo 57}

\date{\today}

%%%%%%%%%%%%%%%%%%%%%%%%%%%%%%%%%% Abstract %%%%%%%%%%%%%%%%%%%%%%%%%%%%%%%%%%%%
\begin{abstract}

\end{abstract}
\maketitle


%%%%%%%%%%%%%%%%%%%%%%%%%%%%%%%%%% Introdução %%%%%%%%%%%%%%%%%%%%%%%%%%%%%%%%%%
\section{Introdução}
\label{s:intro}
%Situar o problema, incluir fórmulas.
%Quem lê o relatório deve conseguir perceber exatamente o que foi feito.

De modo a determinar a impedância de entrada do circuito mede-se a corrente e a tensão de entrada utilizando um multímetro e determina-se a impedância a partir da equação \ref{eq:impedanciaentrada}.
\begin{equation}
Z_{in}=\frac{V_{in}}{I_{in}}
\label{eq:impedanciaentrada}
\end{equation}
Para determinar a impedância de saída do circuito utilizam-se duas cargas diferentes e considera-se que o circuito é linear de modo a poder ser possível considerar um equivalente de Thévenin. A impedância de saída é obtida através da equação \ref{eq:impedanciasaida}.
\begin{equation}
Z_{out}=\frac{R_L \frac{dV_0}{dR_L}}{\frac{V_0}{R_L}-\frac{dV_0}{dR_L}}
\label{eq:impedanciasaida}
\end{equation}

É também pretendido verificar a resposta em frequência do circuito e para tal varia-se a frequência do sinal de entrada e apresenta-se o gráfico de $\frac{i_0}{i_i} (dB)$ em função de $log(f)$. De modo a saber as correntes de entrada e saída medem-se as tensões respectivas e utiliza-se $I=\frac{V}{R}$. Determina-se ainda o limite superior da banda passante a $-3dB$ partir da intersecção dessa recta com o gráfico já referido. 


%%%%%%%%%%%%%%%%%%%%%%%%%%%% Experiência realizada %%%%%%%%%%%%%%%%%%%%%%%%%%%%%
\section{Experiência Realizada}
\label{s:expreal}
%Incluir diagrama de blocos da montagem
De acordo com as instruções do guia, implentou-se o circuito da \rfig{../img/circuito.png}. Apresentam-se na tabela \ref{partlist} os componentes utilizados.

\fig[.4]{../img/circuito.png}{Circuito implementado no laboratório}

\tabela[partlist]{Lista dos componentes utilizados}{lrrr}{
	Descrição		&Modelo/Valor		&Qt.			&Referência	\\ \hline
	Ampl. operacional	&			&1			&		\\
	Cond. cerâmico		&nF			&1			&		\\
	Resistência .25W	&$\Omega$		&4			&		\\
}


%%%%%%%%%%%%%%%%%%%%%%%%%%%%%%%% Resultados %%%%%%%%%%%%%%%%%%%%%%%%%%%%%%%%%%%%
\section{Resultados}
\label{s:resul}
%incluir tabelas. Excesso de dados => grafico com valores mais significativos

\subsection{Ponto de funcionamento em repouso}
Apresentam-se na tabela \ref{PFR} os valores de tensões e correntes relevantes medidos experiementalmente, calculados teoricamente e obtidos na simulação.
\tabela[PFR]{Valores de corrente e tensão, conforme legenda da \rfig{esquema}}{llll}{

}

\subsection{Função de transferência}
\subsection{Potências fornecidas e dissipadas}
\subsection{Impedâncias de entrada e saída}
Neste ponto mediram-se a tensão e corrente de entrada directamente utilizando o multímetro, de modo a determinar a impedância de entrada. Mediram-se também, utilizando duas cargas diferentes, a corrente e tensão nessas cargas, para que, a partir do equivalente de Thévenin fosse possível determinar a impedância de saída do circuito. Os resultados estão na tabela \ref{tab:impedancia}.

\begin{table}[h]
    \begin{tabular}{|l|l|}
\hline
         $ V_i (V)$	&	$0.874  \pm 0.01$		                    \\ \hline
	$I_i (A)$	&	$0.00034 \pm 0.0001$			\\ \hline 
	$R_1$ ($\Omega$) &	$10\pm0.1$			 \\ \hline
	$V_{01}  (V)$	&   	$0.851\pm0.01$	                       \\ \hline
           $R_2$ ($\Omega$) &  $8\pm0.1$   \\ \hline
           $V_{02} (V)$ &   $0.835 \pm 0.01$ \\ \hline

    \end{tabular}
    \caption{Valores obtidos das tensões e correntes de entrada e saída.}
    \label{tab:impedancia}
\end{table}





\subsection{Resposta em frequência}
Variou-se a frequência do sinal de entrada e mediu-se utilizando multímetros a tensão de saída e a tensão de entrada. Os resultados obtidos encontram-se na tabela \ref{tab:respostafrequenciaresultados}
\begin{table}[h]
    \begin{tabular}{|l|l|l|}
    \hline
    Frequência sinal de entrada (Hz) & Vi(V) & Vo(V) \\ \hline
    100                              & 1.20  & 10.6  \\ \hline
    200                              & 1.20  & 10.8  \\ \hline
    400                              & 1.04  & 11.2  \\ \hline
    600                              & 0.96  & 11.4  \\ \hline
    800                              & 1,04  & 11.7  \\ \hline
    1000                             & 0.80  & 12.1  \\ \hline
    2000                             & 0.64  & 12.4  \\ \hline
    4000                             & 0.56  & 12.9  \\ \hline
    6000                             & 0.56  & 12.8  \\ \hline
    8000                             & 0.48  & 13.0  \\ \hline
    10000                            & 0.32  & 13.0  \\ \hline
    20000                            & 0.32  & 12.9  \\ \hline
    40000                            & 0.32  & 13.0  \\ \hline
    60000                            & 0.48  & 12.8  \\ \hline
    80000                            & 0.56  & 12.8  \\ \hline
    100000                           & 0.64  & 12.8  \\ \hline
    200000                           & 2.12  & 12.7  \\ \hline
    400000                           & 2.40  & 12.4  \\ \hline
    600000                           & 3.92  & 13.0  \\ \hline
    800000                           & 4.64  & 13.7  \\ \hline
    1000000                          & 7.04  & 14.1  \\ \hline
    \end{tabular}
\caption{Dados obtidos para a determinação da resposta em frequência do circuito}
\label{tab:respostafrequenciaresultados}
\end{table}








%%%%%%%%%%%%%%%%%%%%%%%%%%% Análise dos resultados %%%%%%%%%%%%%%%%%%%%%%%%%%%%%
\section{Análise de resultados}
\label{s:aresul}
\subsection{Ponto de funcionamento em repouso}
\subsection{Função de transferência}
\subsection{Potências fornecidas e dissipadas}
\subsection{Impedâncias de entrada e saída}
A impedância de entrada foi determinada através dos dados da tabela \ref{tab:impedancia} e da equação \ref{eq:impedanciaentrada}. Para a impedância de saída obtiveram-se 2 valores, sendo que se utilizaram as duas cargas para os cálculos, utilizando a equação \ref{eq:impedanciasaida}. Considerou-se que $\frac{dV_0}{dR_L}=\frac{V_{01}-V_{02}}{R_1-R_2}$ quando se utilizou $R_1$ para os cálculos e $\frac{dV_0}{dR_L}=\frac{V_{02}-V_{01}}{R_2-R_1}$ quando se utilizou $R_2$. Os resultados obtidos encontram-se na tabela \ref{tab:analiseimpedancias}


\begin{table}[h]
    \begin{tabular}{|l|l|l|}
    \hline
    $Z_{in}$($\Omega$) & $2570.59 \pm 78.55$ \\ \hline
    $Z_{01}$($\Omega$) & $1.038\pm 0.751$  \\ \hline
    $Z_{02}$($\Omega$) & $0.664\pm0.009$  \\ \hline
    \end{tabular}
\caption{Resultados obtidos para as impedâncias de entrada e saída}
\label{tab:analiseimpedancias}
\end{table}

Verifica-se, como seria de esperar, que o andar de saída em classe A é um andar com uma alta impedância de entrada e uma baixa impedância de saída.


\subsection{Resposta em frequência}
Tendo em conta os dados da tabela \ref{tab:respostafrequenciadados} e atendendo a que $R_i=70.2 \pm 0.1\Omega$ e $R_o=6.9 \pm 0.1\Omega$ foi possível obter as correntes $i_i$ e $i_o$ a partir de $i=V/R$. Apresenta-se na tabela \ref{tab:respostafrequenciaanalise} $log(f)$ e também $\frac{i_0}{i_i}$ em $dB$. 



%%%%%%%%%%%%%%%%%%%%%%%%%%% Conclusões e Críticas %%%%%%%%%%%%%%%%%%%%%%%%%%%%%%
\section{Conclusões e Críticas}
\label{s:conclu}
%Incluir melhorias propostas à experiência


%\begin{acknowledgments}
%\end{acknowledgments}

%%%%%%%%%%%%%%%%%%%%%%%%%%%%%%%%%%%%%%%%%%%%%%%%%%%%%%%%%%%%%%%%%%%%%%%%%%%%%%%%
% % % % % % % % % % % % % % % %     FIM    % % % % % % % % % % % % % % % % % % % 
%%%%%%%%%%%%%%%%%%%%%%%%%%%%%%%%%%%%%%%%%%%%%%%%%%%%%%%%%%%%%%%%%%%%%%%%%%%%%%%%

\nocite{*}
\bibliography{bibliografia}{}
\bibliographystyle{plain}% Produces the bibliography via BibTeX.
\end{document}
%end of file

% ****** Start of file .tex ******
%
%   This file is based on apssamp.tex, part of the APS files in the REVTeX 4.1 distribution.
%   Version 4.1r of REVTeX, August 2010
%
%   Copyright (c) 2009, 2010 The American Physical Society.
%   Samuel Balula, Pedro Ribeiro, Luís Macedo, Eduardo Neto 2013
%   See the REVTeX 4 README file for restrictions and more information.
%
% TeX'ing this file requires that you have AMS-LaTeX 2.0 installed
% as well as the rest of the prerequisites for REVTeX 4.1
%
% See the REVTeX 4 README file
% It also requires running BibTeX. The commands are as follows:
%
%  1)  latex filename.tex
%  2)  bibtex filename
%  3)  latex filename.tex
%  4)  latex filename.tex

\documentclass[%
  reprint,
  %superscriptaddress,
  %groupedaddress,
  %unsortedaddress,
  %runinaddress,
  %frontmatterverbose, 
  %preprint,
  %showpacs,preprintnumbers,
  nofootinbib,
  %nobibnotes,
  %bibnotes,
  amsmath,amssymb,
  aps,
  %pra,
  %prb,
  %rmp,
  %prstab,
  %prstper,
  %floatfix,
  10pt,
  a4paper
]{revtex4-1}



\usepackage{facil}                      % Pacote pessoal
\usepackage{verbatim}                   % Apresentação de código
\usepackage{graphicx}                   % Include figure files
\usepackage{dcolumn}                    % Align table columns on decimal point
\usepackage{bm}                         % bold math
\usepackage[latin1,utf8]{inputenc}      % Tipos de caracteres
\usepackage[portuges]{babel}            % Português
\usepackage{indentfirst}                % Identação da primeira linha
\usepackage{hyperref}                   % add hypertext capabilities
\usepackage{float}                      %Fixar imagens
%\usepackage[mathlines]{lineno}          % Enable numbering of text and display math
%\linenumbers\relax                      % Commence numbering lines
%\usepackage[compact]{titlesec}

\usepackage[%showframe,%Uncomment any one of the following lines to test 
%%scale=0.7, marginratio={1:1, 2:3}, ignoreall, % default settings
%%text={7in,10in},centering,
margin=0.5in,        %diminuir margens
%total={6.5in,8.75in}, top=1.2in, left=0.9in, 
includefoot
%height=10in,a5paper,hmargin={3cm,0.8in},
]{geometry}

\begin{document}
\preprint{APS/123-QED}
%\captionsetup[table]{font=small,skip=0pt}
%\captionsetup[figure]{font=small,skip=0pt}
%\titlespacing{\section}{0pt}{*0}{*0}            %Poupar espaço
%\titlespacing{\subsection}{0pt}{*0}{*0}
%\titlespacing{\subsubsection}{0pt}{*0}{*0}


% % % % % % % % % % % % % % % % % % % % % % % % % % % % % % % % % % % % % % % % 
%%%%%%%%%%%%%%%%%%%%%%%%%%%%%%%%%% Início %%%%%%%%%%%%%%%%%%%%%%%%%%%%%%%%%%%%%%
% % % % % % % % % % % % % % % % % % % % % % % % % % % % % % % % % % % % % % % %
 

\title{Andar final de amplificação em classe A\\
com transistores bipolares 2n3055}
\thanks{}

\author{Pedro Ribeiro}%
\email{73221, pedro.q.ribeiro@tecnico.ulisboa.pt}
\author{Luis Macedo}%
\email{73633, luis.macedo@tecnico.ulisboa.pt}
\author{Samuel Balula}%
\email{72735, samuel.balula@tecnico.ulisboa.pt}

\affiliation{
  Instituto Superior Técnico\\
  Mestrado em Engenharia Física Tecnológica\\
  Complementos de Electrónica
}

%\collaboration{Grupo 57}

\date{\today}

%%%%%%%%%%%%%%%%%%%%%%%%%%%%%%%%%% Abstract %%%%%%%%%%%%%%%%%%%%%%%%%%%%%%%%%%%%
\begin{abstract}

\end{abstract}
\maketitle


%%%%%%%%%%%%%%%%%%%%%%%%%%%%%%%%%% Introdução %%%%%%%%%%%%%%%%%%%%%%%%%%%%%%%%%%
\section{Introdução}
\label{s:intro}
%Situar o problema, incluir fórmulas.
%Quem lê o relatório deve conseguir perceber exatamente o que foi feito.

\fig[.4]{../img/esquematico.png}{Circuito implementado no laboratório}

Neste trabalho laboratorial implentou-se um andar de saída em classe A, conforme a \rfig{../img/esquematico.png} e que se descreve em maior detalhe em \ref{s:expreal}

\subsection{Ponto de funcionamento em repouso}
Para o cálculo do ponto de funcionamento em repouso utilizaram-se as seguintes relações para os transístores:
\eq{i_C = \beta_f i_b}
\eq{I_C	= I_{ES} \lr{e^{\frac{v_{EB}}{\eta V_T}}}}


%%%%%%%%%%%%%%%%%%%%%%%%%%%% Experiência realizada %%%%%%%%%%%%%%%%%%%%%%%%%%%%%
\section{Experiência Realizada}
\label{s:expreal}
%Incluir diagrama de blocos da montagem
De acordo com as instruções do guia, implentou-se o circuito da \rfig{../img/esquematico.png}. Apresentam-se na tabela \ref{partlist} os componentes utilizados.



\tabela[partlist]{Lista dos componentes utilizados}{lrrr}{
	Descrição		&Modelo/Valor		&Qt.			&Referência	\\ \hline
	Ampl. operacional	&			&1			&		\\
	Cond. cerâmico		&nF			&1			&		\\
	Resistência .25W	&$\Omega$		&4			&		\\
}


%%%%%%%%%%%%%%%%%%%%%%%%%%%%%%%% Resultados %%%%%%%%%%%%%%%%%%%%%%%%%%%%%%%%%%%%
\section{Resultados}
\label{s:resul}
%incluir tabelas. Excesso de dados => grafico com valores mais significativos

\subsection{Ponto de funcionamento em repouso}
Apresentam-se na tabela \ref{PFR} os valores de tensões e correntes relevantes medidos experiementalmente, calculados teoricamente e obtidos na simulação.

\tabela[PFR]{Valores de corrente e tensão, conforme legenda da \rfig{../esquematico.png}}{llll}{
ID	&Experimental		&Teórico	&Simulação	\\ \hline
Vdd (V)	&$8.022\pm0.01$		&$8.000$	&$8.000$	\\
Vss (V)	&$-8.006\pm0.01$	&$-8.000$	&$-8.000$	\\
10  (V)	&$-0.420\pm0.002$	&	&$-0.365$	\\
11  (V)	&$-0.574\pm0.001$	&	&$-0.438$\\
12  (V) &$-1.296\pm0.001$	&	&$-1.149$\\
13  (V) &$-7.253\pm0.001$	&	&$-7.275$\\ \hline
20  (A) &$1.12\pm0.02$		&	&$0.724$\\
21  (A) &$1.88\pm0.02$		&	&$1.574$\\
22  (A) &$1.23\pm0.02$		&$0.7154$	&$0.847$

}

\tabela[Potdiss]{Potências dissipadas no PFR, valores em Watt}{llll}{
ID	&Experimental		&Teórico	&Simulação	\\ \hline
$Vdd$		&$-9.06\pm0.17$		&	&$-5.79$
$Vss$		&$-15.05\pm0.18$	&	&$-12.59$
Ger. sinais	&$-0.0064\pm0.0001$	&	&$-0.0026$
$R_1$		&$0.1680\pm0.0003$	&	&
$R_2$		&$5.2606\pm0.0014$	&	&
$R_3$		&$0.0024\pm0.0001$	&	&
$Q_1$		&$10.53\pm0.20$		&	&
$Q_2$		&$8.25\pm0.15$		&	&
$Q_3$		&$0.489\pm0.037$	&	&



}

\subsection{Função de transferência}
\subsection{Potências fornecidas e dissipadas}
\subsection{Impedâncias de entrada e saída}
\subsection{Resposta em frequência}






%%%%%%%%%%%%%%%%%%%%%%%%%%% Análise dos resultados %%%%%%%%%%%%%%%%%%%%%%%%%%%%%
\section{Análise de resultados}
\label{s:aresul}
\subsection{Ponto de funcionamento em repouso}
\subsection{Função de transferência}
\subsection{Potências fornecidas e dissipadas}
\subsection{Impedâncias de entrada e saída}
\subsection{Resposta em frequência}


%%%%%%%%%%%%%%%%%%%%%%%%%%% Conclusões e Críticas %%%%%%%%%%%%%%%%%%%%%%%%%%%%%%
\section{Conclusões e Críticas}
\label{s:conclu}
%Incluir melhorias propostas à experiência


%\begin{acknowledgments}
%\end{acknowledgments}

%%%%%%%%%%%%%%%%%%%%%%%%%%%%%%%%%%%%%%%%%%%%%%%%%%%%%%%%%%%%%%%%%%%%%%%%%%%%%%%%
% % % % % % % % % % % % % % % %     FIM    % % % % % % % % % % % % % % % % % % % 
%%%%%%%%%%%%%%%%%%%%%%%%%%%%%%%%%%%%%%%%%%%%%%%%%%%%%%%%%%%%%%%%%%%%%%%%%%%%%%%%

\nocite{*}
\bibliography{bibliografia}{}
\bibliographystyle{plain}% Produces the bibliography via BibTeX.
\end{document}
%end of file
